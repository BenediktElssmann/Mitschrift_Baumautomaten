
\documentclass[titlepage]{article}

\usepackage{geometry}
\usepackage[german]{babel}
\usepackage{mathtools}
\usepackage{stmaryrd}
\usepackage{qtree}
\usepackage{amssymb}
\usepackage{amsmath}
\usepackage{tikz}
\usetikzlibrary{automata,positioning}

\setcounter{section}{-1}
\setlength{\parindent}{0pt}


\begin{document}

\title{\huge{Vorlesung Baumautomaten (Mitschrift)}}

\author{Benedikt El\ss mann (3720358)\\
		be57xocu@studserv.uni-leipzig.de\\\\
		Universit\"at Leipzig}

\maketitle

\thispagestyle{empty}

\tableofcontents

\newpage

\section{Einleitung}

Automaten lesen W\"orter $ w = a_1 \dots a_n $ und geben ''accept'' aus oder nicht.\\
Daf\"ur gibt es Erweiterungen, wie etwa:

\begin{itemize}
	\item gewichtete Automaten, das hei\ss t der Output ist ein Semiringelement
	\item Automaten mit Ged\"achtnis (Stack)
	\item Automaten \"uber anderen  Strukturen

	\begin{itemize}
		\item $\omega$-W\"orter $ w = a_1 \dots a_n $ 
		\item Graphen
		\item B\"aume\\
		\item Kombinationen dieser
	\end{itemize}

\end{itemize}

Typische Fragestellungen:

\begin{itemize}
	\item Ausdrucksst\"arke
	\item Darstellung als rationale Ausdr\"ucke (Kleene)
	\item Darstellung als Grammatik
	\item Darstellung als Logik
\end{itemize}

\section{B\"aume und Baumautomaten}

Wir betrachten \"uber $ A = \{a,b\} $ den Automaten $ \mathcal{A} $:

\begin{tikzpicture}[shorten >= 1pt, node distance=2cm, on grid, auto]
	\node[state, initial] (q0) {$ q_0 $};
	\node[state] (q1) [right=of q0 ] {$ q_1 $};
	\node[state, accepting] (q2) [right=of q1 ] {$ q_2 $};

	\path[->]	(q0)	edge 					node	{a}	(q1);
	\path[->]	(q0)	edge 	[loop above] 	node 	{b}	();
	\path[->]	(q1)	edge					node 	{b}	(q2);
	\path[->]	(q2)	edge	[loop above]	node	{a}	();
\end{tikzpicture}

mit $ L(\mathcal{A}) = b^\ast ab a^\ast $ .

Betrachtung des Wortes $ w = baba \in L(\mathcal{A}) $:\\

Der eindeutige erfolgreiche Lauf f\"ur $ w $ l\"asst sich darstellen als:\\

$q_0 baba \rightarrow b q_0 aba \rightarrow ba q_1 ba \rightarrow bab q_2 a \rightarrow 
baba q_2 \in F$ (Finalzustand)\\

Baumautomaten funktionieren analog. Unser erstes Beispiel wird

\Tree [.f [.f a b ] [.b ] ]

Akzeptiert mit dem Lauf:

\Tree [.f [.f a b ] [.b ] ] $\rightarrow$
\Tree [.f [.f $q_a$ $q_b$ ] [.$q_a$ ] ] $\rightarrow$
\Tree [.f [.$q_a$ $q_a$ $q_b$ ] [.$q_a$ ] ] $\rightarrow$
\Tree [.$q_f$ [.$q_a$ $q_a$ $q_b$ ] [.$q_a$ ] ]\\

mit $q_f \in F$ 

\subsection{Definition Rangalphabet}

Ein paar $(\Sigma, rk)$, wobei $\Sigma$ eine endliche Menge von Symbolen und 
$rk : \Sigma \rightarrow \mathbb{N}$ eine Abbildung ist, hei\ss t Rangalphabet.\\
F\"ur $f\in \Sigma$ hei\ss t $rk(f)$ der Rang (oder die Stelligkeit) von $f$.\\ \\
Intuitiv: $rk(f)$ ist die Anzhal der Kinder von $f$ in einem Baum.\\
Insbesondere ist die Anzahl der Kinder f\"ur jedes Symbol fest.\\ \\
Gilt $rk(f) = n$, schreiben wir auch $f^{(n)}$ statt $f$. wir schreiben:

\begin{itemize}
	\item 0-stellige Symbole (Konstanten) $a, b, \dots$
	\item un\"ar, bin\"ar, \dots $f, g, \dots$\\
\end{itemize}

Wir setzen $\Sigma^{(n)} = \{f \in \Sigma | rk(f) = n\}$\\

In 
\Tree [.f [.f a b ] [.b ] ] 
ist also $rk(f) = 2, rk(b) = 0$\\

\subsection{Definition Term, Tree}

Sei $(\Sigma, rk)$ ein Rangalphabet.
Die Menge $T_{\Sigma}$ der B\"aume \"ueber $\Sigma$ ist induktiv definiert durch:

\begin{itemize}
	\item $\Sigma^0 \subseteq T_{\Sigma}$
	\item $f^{(n)} \in \Sigma$ . $t_1, \dots, t_n \in T_{\Sigma}$, 
		dann ist $f(t_1, \dots, t_n) \in T_{\Sigma}$
\end{itemize}

Intuitiv sind $t_1, \dots, t_n$ die Kinder von $f$.\\
Z.B. ist \Tree [.f [.f a b ] [.b ] ] der Term $f(f(a,b),b)$.

\subsection{Definition H\"ohe}

Sei $(\Sigma, rk)$ ein Rangalphabet. Die H\"ohe $ht$ ist gegeben durch:

\begin{itemize}
	\item f\"ur $a^{(0)} \in \Sigma: ht(a) = 1$.
	\item f\"ur $f(t_1, \dots , t_n) \in T_{\Sigma} : 
		ht(f) = 1 + max\{ ht(t_i) | i \in \{i, \dots, n\}\}$
\end{itemize}

Ziel: Zugriff auf einen Knoten innterhalb eines Baumes und deren Label.\\
Daf\"ur ordnen wir den Knoten Positionen zu. Das geht induktiv wie folgt:

\subsection{Definition Position}

Sei $(\Sigma, rk)$ ein Rangalphabet. Die Positionenmenge ist definiert durch:

\begin{itemize}
	\item f\"ur $a^{(0)} \in T_{\Sigma}$ ist $Pos(a) = \{\varepsilon\}$
	\item f\"ur $f(t_1, \dots, t_n) \in T_{\Sigma}$ ist 
	$Pos(f(t_1, \dots, t_n)) = \{\varepsilon\} \cup 1 \cdot Pos(t_1) \cup 
		\dots \cup n \cdot Pos(t_n)$
\end{itemize}

Beispiel:\\
Betrachtung von $f(f(a,b),b)$ bzw. \Tree [.f [.f a b ] [.b ] ] der Term $f(f(a,b),b)$\\

$Pos(f) = \{\varepsilon, 1, 2, 1.1, 1.2\}$

\subsection{Definition der Label an den Positionen}

F\"ur einen Term der Form $t = f(t_1, \dots, t_n)$ ist das Symbol $t(p)$ in $t$ an p-ter 
Position induktiv definert durch:

\begin{itemize}
	\item $t(\varepsilon) = f$
	\item $t(ip) = t_i(p), i\in \{1, \dots , n\}$
\end{itemize}

Beispiel: Betrachtung von $f(f(a,b),b)$\\
Dann ist\\
$t(\varepsilon) = f$\\
$t(1) = t(1 \cdot \varepsilon) = t_1(\varepsilon) = f$\\
$t(2) = t(2 \cdot \varepsilon) = t_2(\varepsilon) = b$\\
$t(1.1) = t_1(1) = a$\\
$t(1.2) = t_2(1) = b$

\subsection{Definition Sub-Baum}

F\"ur $T_{\Sigma}$ ist ein Sub-Baum $t_{|p}$ an p-ter Position wie folgt definiert:

\begin{itemize}
	\item $Pos(t_{|p}) = \{ i | pi \ \in Pos(t)\}$
	\item $\forall q \in Pos(t_{|p}$ ist $t_{|p}(q) = t(pq)$
\end{itemize}

Wir schreiben $t[u]_p$ f\"ur den Baum, der entsteht, wenn man in $t$ den sub-Baum 
$t_{|p}$ durch $n$ ersetzt.\\

Beispiel: $f(f(a,b),b)$ bzw. \Tree [.f [.f a b ] [.b ] ]

$t_{|1} = f(a,b)$ \Tree [.f a b ]\\

$t_{|2} = t(1.2) = b$\\

$u = g(b,a)$ \Tree [.g b a ], dann ist\\

$t[u]_{|1} = f(g(b,a),b)$ \Tree [.f [.g b a ] [.b ] ]

\subsection{Definition Baumautomat}

Ein Baumautomat $ \mathcal{A}$ ist ein 4-Tupel $(Q,\Sigma,F,\Delta)$, wobei:\\
$Q \dots$ endliche Menge an Zus\"anden\\
$\Sigma \dots$ Rangalphabet, wobei $\Sigma \cup Q \neq \emptyset$\\
$F \dots \subseteq Q$ Finalzust\"ande\\
$\Delta \dots$ Menge von Regeln\\

$r: f(q_1 \dots q_n) \rightarrow q$\\
f\"ur $q, q_1, \dots , q_n \in Q$, f\"ur $a^{(0)} \in T_{\Sigma} : a \rightarrow q$\\

Beispiel:\\
$\mathcal{A} = \{\{q_a, q_b, q_f\},\{a^{(0)},b^{(0)},f^{(2)}\},\{q_f\}, \Delta\}$ \\
mit $\Delta = \{ a \rightarrow q_a, b \rightarrow q_b, f(q_a,q_b) \rightarrow q_a, 
f(q_a,q_b) \rightarrow q_f\}$

\subsection{Definition Lauf/Run}

Sei $\mathcal{A} = (Q, \Sigma, F, \Delta)$ ein Baumautomat und $t\in T_{\Sigma}$. 
Ein Lauf $r$ f\"ur $t$ von $\mathcal{A}$ ist ein Term mit

\begin{itemize}
	\item $Pos(r) = Pos(t)$
	\item Ist $t(p) = a$ ein Blatt, dann ist $r(p) = q_a$, nur wenn 
		$(a\rightarrow q_a) \in \Delta$ 
	\item Ist $t(p)=f^{(m)}$, dann ist $r(p) = q$, wenn 
		$(f(q_1, \dots, q_n) \rightarrow q) \in \Delta$ und 
		$r(p_i) = q_i$, $i \in \{1, \dots, n\}$
\end{itemize}

Ein Lauf ist erfolgreich, wenn $r(\varepsilon) \in F$. 
Der Automat $\mathcal{A}$ akzeptiert $t$, falls es einen erfolgreichen Lauf
f\"ur $t$ von $\mathcal{A}$ gibt.\\
Wir bezeichnen mit $L(\mathcal{A}) = \{t \in T_{\Sigma} | \mathcal{A}$ akzeptiert $t\}$ 
die von $\mathcal{A}$ erkannte Baumsprache. 
Eine Sprache $L \subseteq T_{\Sigma}$ hei\ss t erkennbar, falls ein Baumautomat 
$\mathcal{A}$ existiert mit $L=L(\mathcal{A})$.\\

Um einzelne Schritte von Baumautomaten zu formalisieren, betrachten wir die 
\textit{move relation} $\rightarrow_{\mathcal{A}}$, 
definiert wie folgt:\\
Gegeben sei $\mathcal{A} = (Q, \Sigma, F, \Delta)$, dann ist 
$t \rightarrow _{\mathcal{A}} t'$ mit $t, t' \in T_{\Sigma \cup Q}$, falls

\begin{itemize}
	\item $t(p) = f^{(n)}$
	\item $t(pi) = q_i$ f\"ur $i \in \{1, \dots, n\}$ und $p_i$ sind Bl\"atter
	\item $(f(q_1, \dots, q_n) \rightarrow q) \in \Delta$
	\item und $t' = t[q]_p$
\end{itemize}

Mit $\rightarrow^\ast_{\mathcal{A}}$ bezeichnen wir die transitive H\"ulle von 
$\rightarrow_{\mathcal{A}}$.

\subsection{Lemma}

Sei $\mathcal{A} = (Q, \Sigma, F, \Delta)$ ein Baumautomat. Dann ist 
$L(\mathcal{A}) = \{t \in T_{\Sigma} | t \rightarrow^\ast_{\mathcal{A}} q$ mit 
$q \in F\} (=Z)$\\

Beweis: \glqq$L(\mathcal{A}) \subseteq Z$\grqq:\\

Wir zeigen: Es existiert ein Run $r$ f\"ur $t$ von $\mathcal{A}$ mit 
$r(\varepsilon) = q$, dann ist $t \rightarrow ^\ast_{\mathcal{A}} q$\\ \\
Inuktionsannahme:\\
$t=a^{(0)} \in T_{\Sigma}$. Dann gilt $a \in L({\mathcal{A}})$, falls ein Lauf $r$ 
existiert mit $r(a) = q_a$ und $(a \rightarrow q_a) \in \Delta$. 
Dann folgt $a \rightarrow^\ast_{{\mathcal{A}}} q_a$.\\
Sei nun $t=f(t_1, \dots, t_n)$\\ \\

Induktionsvoraussetzung:\\
Falls f\"ur $t_1, \dots , t_n$ L\"aufe $r_i$ existieren mit $r_i(\varepsilon) = q_i$, 
dann gilt auch $t_i \rightarrow ^\ast_{\mathcal{A}} q_i$ mit $i\in \{1, \dots , n\}$\\ \\

Induktionsschritt:\\
zu zeigen: Es existiert ein Lauf $r$ f\"ur $t$ mit $r(\varepsilon) = q$, dann 
$t \rightarrow ^\ast_\mathcal{A} q$.\\
Sei also $r$ ein Lauf mit $r(\varepsilon) = q$. Dann ist $r(i) = q_i$, 
$i \in \{1, \dots, n\}$, mit
$(f(q_1, \dots, q_n) \rightarrow q) \in \Delta$. Laut Induktionsvoraussetzung gilt nun,
$t_i \rightarrow ^\ast_\mathcal{A} q_i$, $i \in \{1, \dots, n\}$.\\
Damit $t = f(t_1, \dots, t_n) 
\rightarrow ^\ast_\mathcal{A} f(q_1, t_2, \dots, t_n)
\rightarrow ^\ast_\mathcal{A} \dots
\rightarrow ^\ast_\mathcal{A} f(q_1, \dots, q_n)$\\
Des weiteren haben wir die regel $f(q_1, \dots, q_n) \rightarrow q$, das hei\ss t 
$f(q_1, \dots, q_n) \rightarrow ^\ast_\mathcal{A} q$.\\ \\

Insgesamt also $t \rightarrow ^\ast_\mathcal{A} q$ \\

Beweis: \glqq$L(Z \subseteq \mathcal{A})$\grqq: analog\\ \\

Einige Beispiele f\"ur Baumautomaten:\\

1. Sei $B = (\{q_0, q_1\}, \{0^{(0)},1^{(0)},\lnot^{(1)},\land^{(2)},\lor^{(2)}\,
\{q_1\},\Delta\}$ mit\\
$\Delta = \{ 0 \rightarrow q_0, 1 \rightarrow q_1,\\
\lnot (q_0) \rightarrow q_1, \lnot (q_1) \rightarrow q_0,\\
\land (q_0, q_0) \rightarrow q_0, \land (q_0, q_1) \rightarrow q_0, 
\land (q_1, q_0) \rightarrow q_0, \land (q_1, q_1) \rightarrow q_1\\
\lor (q_0, q_0) \rightarrow q_0, \lor (q_0, q_1) \rightarrow q_1, 
\lor (q_1, q_0) \rightarrow q_1, \lor (q_1, q_1) \rightarrow q_1\}$\\

Beispiellauf:\\

\Tree [.$\lnot$ [.$\land$ [.$\lor$ 0 [.$\lnot$ 1 ] ] [.$\lor$ 1 [.$\land$ 0 0 ] ] ] ] 
	$\rightarrow$
\Tree [.$\lnot$ [.$\land$ [.$\lor$ $q_0$ [.$\lnot$ $q_1$ ] ] 
[.$\lor$ $q_1$ [.$\land$ $q_0$ $q_0$ ] ] ] ] 
	$\rightarrow$
\Tree [.$\lnot$ [.$\land$ [.$\lor$ $q_0$ [.$q_0$ ] ] 
[.$\lor$ $q_1$ [.$\land$ $q_0$ $q_0$ ] ] ] ] 
	$\rightarrow$ \\ \\
\Tree [.$\lnot$ [.$\land$ [.$\lor$ $q_0$ $q_0$ ] [.$\lor$ $q_1$ $q_0$ ] ] ] 
	$\rightarrow$
\Tree [.$\lnot$ [.$\land$ $q_0$ $q_1$ ] ] 
	$\rightarrow$
\Tree [.$\lnot$ $q_0$ ] 
	$\rightarrow$
\Tree [.$q_1$ ] \\ \\

2. $(a^nb^n light)$\\

Betrachten 
$\mathcal{A} = (\{q_a, q_b, q_f\}, \{a^{(0)}, b^{(0)}, f^{(3)}, g^{(2)}\}, \{q_f\}, 
\Delta)$\\
mit $\Delta = \\
\{a \rightarrow q_a, b \rightarrow q_b, g (q_a, q_b) \rightarrow q_f, f (q_a, 
q_f, q_b) \rightarrow q_f\}$\\

Beispiellauf:\\

\Tree [.$f$ $a$ [.$f$ $a$ [.$g$ $a$ $b$ ] $b$ ] $b$ ] $\rightarrow$
\Tree [.$f$ $q_a$ [.$f$ $q_a$ [.$g$ $q_a$ $q_b$ ] $q_b$ ] $q_b$ ] $\rightarrow$
\Tree [.$f$ $q_a$ [.$f$ $q_a$ $q_f$ $q_b$ ] $q_b$ ] $\rightarrow$
\Tree [.$f$ $q_a$ $q_f$ $q_b$ ] $\rightarrow$
\Tree [.$q_f$ ] \\

$\mathcal{A}$ akzeptiert also alle B\"aume der Form:\\
\Tree [.$f$ $a$ [.$f$ $a$ [.... [.$g$ $a$ $b$ ] ] $b$ ] $b$ ]\\

3. Simulation eines Wortautomaten: (siehe \"Ubung)\\

Betrachtet man $\Sigma =\{a^{(0)}, f^{(2)}, g^{(1)}\}$. 
Dann ist $L = \{f(g^i(a), g^i(a)) | i \geq 0\}$ nicht erkennbar.

\subsection{Definition Determinismus}

Ein Automat $\mathcal{A} (Q, \Sigma,  F, \Delta)$ hei\ss t deterministisch, falls: aus
$ f(q_1, \dots, q_n) \rightarrow q $ und \\
$ f(q_1, \dots, q_n) \rightarrow q' $ folgt $ q = q'$

\subsection{Satz}

Sei $\mathcal{A} = (Q, \Sigma, F, \Delta)$ ein Baumautomat, dann existiert ein 
deterministischer Baumautomat $\mathcal{A}_d$, 
so dass $L(\mathcal{A}) = L(\mathcal{A}_d)$.\\ \\

Beweis: Setze $\mathcal{A}_d = (Q_d, \Sigma, F_d, \Delta_d)$ \\
mit $Q_d = 2^Q$ $(\ast)$\\
und $f(s_1, \dots, s_n) \rightarrow s \in \Delta_d$\\
$\Leftrightarrow s = \{ q \in Q | \exists q_1 \in s_1 \dots q_n \in s_n: 
(f(q_1, \dots, q_n) \rightarrow q) \in \Delta\}$ \\
und $F_d = \{ s \in Q_d | s \cap F \neq \emptyset\}$.\\ \\
Wir zeigen:\\
1. $\mathcal{A}$ ist deterministisch \\
2. $L(\mathcal{A}) \subset L(\mathcal{A}_d)$ \\
3. $L(\mathcal{A}_d) \subset L(\mathcal{A})$ \\ \\

1. ist klar, denn $(\ast)$ ist mit einer \"Aquivalenz definiert.\\
2. \glqq$L(\mathcal{A}) \subseteq L(\mathcal{A}_d)$\grqq:\\

Wir zeigen hierzu: Ist $Z = \{ q | t \rightarrow ^\ast_\mathcal{A} q\}$, 
dann $ t \rightarrow ^\ast_{\mathcal{A}_d} z$.\\ \\

Induktionsannahme:\\
Angenommen $a \rightarrow _\mathcal{A} q_a$, dann ist 
$q_a \in \{ q \in Q | q \rightarrow ^\ast_\mathcal{A} q\}$, das hei\ss t\\
$a \rightarrow ^\ast_\mathcal{A} q_a \Leftrightarrow q_a \in 
\{ q \in Q | a \rightarrow ^\ast_\mathcal{A} q \}$\\
$\Leftrightarrow q_a \in \{ q \in Q | (a \rightarrow q) \in \Delta \}$\\
also $a \rightarrow ^\ast_\mathcal{A} q_a \Leftrightarrow q_a \in 
\{ q \in Q | (a \rightarrow q) \in \Delta \}$, das hei\ss t\\
$z := \{ q_a \in Q | a \rightarrow ^\ast_\mathcal{A} q_a \} = 
\{ q \in Q | (a \rightarrow q) \in \Delta \} =: s$\\

Nun ist $( a \rightarrow s) \in \Delta_d$ per Definition, also auch 
$( a \rightarrow z) \in \Delta_d$, damit: $a \rightarrow ^\ast_{\mathcal{A}_d} z$.\\

Betrachten wir nun $ t = \sigma(t_1, \dots , t_n) $ \\

Induktionsvoraussetzung:\\
$t_i \rightarrow ^\ast_{\mathcal{A}_d} z_i$ mit $Z_i = 
\{q \in Q | t_i \rightarrow ^\ast_\mathcal{A} q\}$\\
Das hei\ss t, es existieren L\"aufe $r_i$ f\"ur $t_i$ von $\mathcal{A}_d$ mit 
$r_i(\varepsilon) = z_i$\\

Induktionsschritt:\\
zu zeigen: $ t \rightarrow ^\ast_\mathcal{A} z$ mit $Z = 
\{ q \in Q | t \rightarrow ^\ast_\mathcal{A} q \} $ \\
Das hei\ss t, es existiert ein Lauf $r$ f\"ur $t$ von $\mathcal{A}_d$ mit 
$r(\varepsilon) = z$\\
Das hei\ss t, $\exists r :$\\

\begin{itemize}
	\item $r (\varepsilon) = z$
	\item $r (i) = z_i$
	\item $\sigma (z_1, \dots, z_n) \rightarrow z \in \Delta_d$
\end{itemize}

Setze nun $r_{ |i } = r_i$, damit ist insbesondere 
$r(i) = r_i(\varepsilon) = z := \{q | t_i \rightarrow ^\ast_{\mathcal{A}_d} q \}$\\

Es bleibt also zu zeigen: $ \exists$ Regel $\sigma (z_i, \dots, z_m) \rightarrow z 
\in \Delta_d$.\\
Es ist nun $ z \in Z \Leftrightarrow t \rightarrow ^\ast_\mathcal{A} z$ \\
$\Leftrightarrow \exists q_i \in Q: t_i \rightarrow ^\ast_\mathcal{A} q_i, 
\sigma(q_1, \dots, q_m) \rightarrow z \in \Delta$\\
$\Leftrightarrow \exists z_i \in Z_i$ und $\sigma/z_1, \dots , z_m) \rightarrow z 
\in \Delta$ \\
Also $Z= \{z\in Q | \exists z_i \in Z_i : (\sigma/z_1, \dots , z_m) \rightarrow z ) 
\in \Delta)$ also per Definition
$\sigma/z_1, \dots , z_m) \rightarrow z \in \Delta_d$\\ \\

2. \glqq$L(\mathcal{A}_d) \subseteq L(\mathcal{A})$\grqq:\\

Sei $t \in T_\Sigma$ mit $t \notin L(\mathcal{A})$, 
dann ist $Z \cap F = \{ q \in Q | t \rightarrow^\ast_\mathcal{A} q \} \cap F = \emptyset$\\
Laut 2. ist $t \rightarrow ^\ast_{\mathcal{A}_d} z$ (und $\mathcal{A}$ ist deterministisch)
Wegen $Z \cap F = \emptyset$ ist $Z \notin F_d$, also $t \notin L(\mathcal{A}_d)$ \\ \\

Wir vereinbaren die Abk\"urzungen: NBA/NTA f\"ur nichtdeterministischer Baumautomat 
und DBA/DTA f\"ur deterministischer Baumautomat.\\
Wie im Wortfall ist die Konstruktion exponentiell, das hei\ss t wir ben\"otigen 
expontntiell viele Zust\"ande ($Q_d = 2^{|Q|}$).
Und wie im Wortfall l\"asst sich das im Allgemeinen nicht vermeiden.\\ \\
Beispiel: Betrachtet man $\Sigma = \{ f^{(1)}, g^{(1)}, a^{(0)}\}$ und
sei $L_n = \{ f \in T_\Sigma | t( \underbrace{1 \dots 1}_\text{n-mal} ) = f \}$ \\

Ein NTA ben\"otigt $n+2$ Zust\"ande:\\
$\mathcal{A} = (Q, \Sigma, F, \Delta)$ mit $Q = \{q, q_1, \dots , q_{n+1} \}$, 
$F = \{q_{n+1}\}$ \\ mit \"Uberg\"angen
$\Delta = \{a \rightarrow q, f(q) \rightarrow q, g(q) \rightarrow q, f(q) \rightarrow q_1, \\
f(q_i) \rightarrow q_{i+1}, g(q_i) \rightarrow q_{i+1} \}$ f\"ur $i \in \{1, \dots ,n\}$\\

Man kann zeigen: Ein DTA $\mathcal{A}'$ mit $L(\mathcal{A}') = L_n$ hat mindestens 
$2^{n+1}$ Zust\"ande.

\subsection{Definition vollst\"anding und reduziert}

Ein Automat $(\mathcal{A} = Q, \Sigma, F, \Delta)$ hei\ss t:

\begin{itemize}
	\item vollst\"andig, falls f\"ur jedes $f^{(n)} \in \Sigma$ und alle 
		$q_1, \dots , q_n \in Q$ eine Regel 
		$f(q_1, \dots, q_n) \rightarrow q \in \Delta$ existiert.
	\item reduziert, falls f\"ur jeden Zustand $q \in Q$ ein Term $t \in T_\Sigma$ 
		exisitert mit $f \rightarrow ^\ast_\mathcal{A} q$
\end{itemize}

\subsection{Satz}

Sei $\mathcal{A}$ ein Baumautomat. Dann existiert ein vollst\"andiger, reduzierter 
Baumautomat $\mathcal{A}'$ mit $L(\mathcal{A}) = L(\mathcal{A}')$.\\

F\"ur Wortautomaten gibt es das Pumping-Lemma, das die Ged\"achtnislosigkeit der 
Automaten formalisiert.
Formal besagt es: Ist $L$ eine regul\"are Wortsprache, dann existiert ein 
$n \in \mathbb{N}$, so dass sich $w \in L$ mit $|w| > n$ 
zerlegen l\"asst in $w = xyz$, $y \neq \varepsilon$ und 
$\forall i \geq 0$ ist $xy^iz \in L$. \\ \\

Baumautomaten haben auch kein Ged\"achtnis, also erwarten wir ein analoges Resultat.
Dazu m\"ussen wir formalisierten, was \glqq aufgepumpt \grqq werden soll.

\subsection{Definition Kontext}

Es sei $\Sigma$ ein Rangalphabet und $x^{(0)} \notin \Sigma$. Es sei 
$C \in T_{\Sigma \cup \{x\}}$.
Falls es genau eine Position $p \in Pos(C)$ gibt mit $C(p) = x$, dann hei\ss t $C$ 
ein Kontext.\\

Beispiel: \Tree [.f [.f x b ] [.b ] ] ist ein Kontext.\\

Wir schreiben $T_\Sigma(x)$ f\"ur die Menge aller solcher Kontexte.\\

Ist $C \in T_\Sigma(x)$ mit $C(p) = x$, dann schreiben wir $C[u]$ statt $C[u]_p$ 
f\"ur den Baum, der entsteht, wenn wir $x$ durch $u$ ersetzen.\\
Wir schreiben $C^0 = x$, $C^1 = C$, $C^n = C^{n-1} [C]$\\

Beispiel: Betrachtet $t = $ \Tree [.f [.f a b ] [.b ] ]\\

Setze $u = f(a,b)$ und $C = f(x,b)$.\\
Dann ist $t = C[u]$ und $C^2[u] = $
\Tree [.f [.f [.f a b ] b ] [.b ] ]

\subsection{Pumping-Lemma}

Sei $L \subseteq t_\Sigma$ erkennbar, dann existiert ein $k \in \mathbb{N}$, so dass:\\
F\"ur alle $T \in L$ mit $ht(t) > k$ gibt es einen Kontext $C \in T_\Sigma (x)$, einen 
nicht-trivialen Kontext $C' \in T_\Sigma (x)$
und einen Term $u \in T_\Sigma$ mit $t = C[C'[u]]$ und 
$C[(C')^n[u]] \in L$ f\"ur alle $n \ge 0$.\\ \\

Beweis: Sei $L$ erkennbar, das hei\ss t $\exists$ Baumautomat 
$\mathcal{A} = (Q, \Sigma, F, \Delta)$ mit $L = L(\mathcal{A})$.
Setze $|Q| = k$ und betrachte $t \in L$ mit $ht(t) > k$. 
Betrachte nun einen Lauf $r$ und einen Pfad in $t$, der l\"anger als $k$ ist.
Nun gibt es $p_1,p_2 \in Pos(r)$ mit $r(p_1) = r(p_2) = q \in Q$.
Sei nun $u = t_{|p_2}$ der Sub-Baum von $t$ bei $p_2$ und $u' = t_{|p_1}$.
Dann existiert $C'$ mit $C'[u] = u'$ und es existiert $C$ mit $t=C[C'[u]]$.
Es ist wegen $t \in L$\\
$C[C'[u]] \rightarrow ^\ast_\mathcal{A} C[C'[q]] \rightarrow ^\ast_\mathcal{A} C[q] 
\rightarrow ^\ast_\mathcal{A} q_f \in F$, also auch\\
$C[(C')^n[u]] \rightarrow ^\ast_\mathcal{A} C[(C')^n[q]] 
\rightarrow ^\ast_\mathcal{A} CC[(C')^{(n-1)}[q]] 
\rightarrow ^\ast_\mathcal{A} \dots \rightarrow ^\ast_\mathcal{A} C[q] 
\rightarrow ^\ast_\mathcal{A} q_f \in F$. q.e.d.\\ \\

Beispiel: Betrachte den Baumautomaten 
$\mathcal{A} = (\{q_a, q_b, q_g, q_f\}, \{a^{(0)}, b^{(0)}, g^{(1)}, f^{(2)}\}, \{q_f\}, 
\Delta)$ mit\\
$\Delta = \{ a \rightarrow q_a, b \rightarrow q_b, f(q_a, q_b) \rightarrow q_g, g(q_g) 
\rightarrow q_f\}$ \\

\Tree [.g [.f [.f [.f a b ] b ] b ] ] \\
$u = f(a, b)$, $u' = C'[u] = f(f(a,b),b)$ \\
$C = g(f(x, b))$, $C' = f(x, b)$ \\

$C[(C')^n[u]] = $ \Tree [.g [.f [.f [.... [.f a b ] ] b ] b ] ] \\

Die Sprache $L = \{ f(g^i(a),g^i(a)) | i \geq 0\}$ kann nicht erkennbar sein, 
denn f\"ur gro\ss e $i$ w\"urde man ein $k$ finden,
so dass ein gegebener Baumautomat auch $f(g^{i+lk}(a), g^i(a))$ f\"ur alle $l \geq 0$ 
akzeptiert.

\subsection{Korollar}

F\"ur $\mathcal{A} = (Q, \Sigma, F, \Delta)$ ist 
$L(\mathcal{A}) \neq \emptyset \Leftrightarrow \exists t \in L$ mit $ht(t) \leq |Q|$:

\begin{itemize}
	\item $L(\mathcal{A}) |$ nicht endlich 
		$\Leftrightarrow \exists t \in L$ mit $|Q| < ht(t) \leq 2 |Q|$
\end{itemize}

\subsection{Abschlusseigenschaften}

Erkennbare Sprachen sind abgeschlossen unter Vereinigung, Schnitt und Komplement. 
Das hei\ss t, sind $L_1$ und $L_2$ erkennbar, dann auch
$L_1 \cup L_2$, $L_1 \cap L_2$ und $L_1^c$ (in $T_\Sigma$).\\

Beweis:\\
Seien $\mathcal{A}_1$ und $\mathcal{A}_2$ vollst\"andige DTA. 
Betrachte f\"ur die Vereinigung\\
$\mathcal{A}_\cup = (Q_1 \times Q_2, \Sigma, F_1 \times Q_2 \cup Q_1 \times F_2, 
\Delta_1 \times \Delta_2)$ mit \\
$\Delta_1 \times \Delta_2 = \{ f((q_1, q'_1), \dots, (q_n, q'_n)) \rightarrow (q, q') |
f(q_1, \dots, q_n) \rightarrow q \in \Delta_1, f(q'_1, \dots, q'_n) \rightarrow q' \in 
\Delta_2 \}$

Dann akzeptiert $\mathcal{A}_\cup$ die Sprache $L(\mathcal{A}_1) \cup L(\mathcal{A}_2)$.\\ \\

F\"ur $L(\mathcal{A}_1) \cap L(\mathcal{A}_2)$ betrachte den Automaten

$\mathcal{A}_\cap = (Q_1 \times Q_2, \Sigma, F_1 \times F_2, \Delta_1 \times \Delta_2)$\\

F\"ur $T_\Sigma \ L(\mathcal{A}_1) = L(A_1)^c$ betrachte

$\mathcal{A}_C = (Q_1, \Sigma, Q_1 \ F_1, \Delta_1)$.

Der Automat $A_C$ akzeptiert $L(A_1)^c$.\\

Beispiel:\\

Betrachte $\Sigma = \{ a^{(0)}, g^{(1)}, f^{(2)}\}$ und 
$L = \{ f(g^i(a),g^j(a)) | i \leq j \}$

Dann ist $L$ nicht erkennbar, denn: W\"are $L$ erkennbar, dann auch $L'$ mit 
$L' = \{ f(g^i(a), g^j(a)) | i \geq j \}$, 
also auch $L \cap L' \lightning$. \\

Bemerkung: Wenn $\mathcal{A}$ deterministisch und vollst\"andig ist, 
dann k\"onnen wir eine \"Ubergangsfunktion 
$\delta: T_\Sigma \rightarrow Q$ definieren mit $\delta (t) = q$, falls 
$t \rightarrow_\mathcal{A}^\ast q$.\\

Wiederholung - \"Aquivalenzrelation:\\

Eine \"Aquivalenzrelation $\sim $ auf einer Menge $M$ ist eine Relation mit

\begin{itemize}
	\item $\forall m \in M: m\sim m$
	\item $\forall m, n \in M: m\sim n \Rightarrow n\sim m$
	\item $\forall l, m, n \in M: l\sim m, m\sim n \Rightarrow l\sim n$
\end{itemize}

Insbesondere: Ist $\sim $ eine \"Aquivalenzrelation auf $M$, so induziert $\sim $ eine 
Partition auf und umgekehrt, das hei\ss t Mengen 
$(M_i)_{i \in I}$ mit $M_i \cup M_j = \emptyset$ f\"ur $i \neq j$ und 
$M = \underset{i \in I}{\cup} M_i$

\subsection{Definition Kongruenz}

Eine \"Aquivalenzrelation $\equiv$ auf $T_\Sigma$ hei\ss t Kongruenz, 
falls f\"ur alle $f^{(n)} \in \Sigma$: \\

$v_1 \equiv u_1, \dots, v_n \equiv u_n \Rightarrow f(v_1, \dots, v_n) \equiv 
f(u_1, \dots, u_n)$.

Beispiel:\\
Die Relation $t \equiv t'$, falls $t$ und $t'$ die gleiche Anzahl Bl\"atter modulo 2 haben.

\begin{itemize}
	\item Au\ss erdem: $t \equiv t' \Leftrightarrow ht(t) = ht(t')$
	\item Nicht: gleiche H\"ohe modulo 2
\end{itemize}

\subsection{Definition}

Eine Kongruenz $\equiv$ hat endlichen Index, falls $\equiv$ endlich viele 
\"Aquivalenzklassen indiziert.

\subsection{Lemma}

Sei $\Sigma$ ein Rangalphabet. Dann ist $\equiv$ genau dann eine Konguenz auf $T_\Sigma$, 
wenn $\equiv$ eine \"Aquivalenzrelation ist mit
$u \equiv v \Rightarrow C[u] \equiv C[v]$ f\"ur alle Kontexte.\\

Beweis:\\
\glqq $\Rightarrow$ \grqq Induktion:\\
Induktionsannahme: $C = x$, dann ist $u \equiv v \Rightarrow C[u] \equiv C[v]$ klar.\\
Sei nun $C = f(C_1, \dots C_n)$. Sei $x = C[ip] = C_i[p]$.\\
Dann ist $C[u]_{ip} = f(C_1, \dots, C_{i-1}, C_i[p], C_{i+1}, \dots, C_n) = C[u]_{ip} = 
f(C_1, \dots, C_{i-1}, C_i[v], C_{i+1}, \dots, C_n)$
\\

\glqq $\Leftarrow$ \grqq: Angenommen $u \equiv v$ und $C[u] \equiv C[v]$ f\"ur alle 
Kontexte.\\
Sei $f^{(n)} \in \Sigma$. Dann ist:\\
$f(u_1, \dots u_n)\\
= C^1[u_1] \equiv C^1[v_1] = f(v_1, u_2, \dots, u_n)\\
= \dots\\
= C^1[u_n] \equiv C^1[v_n] = f(v_1, \dots, v_n)$

Betrachte nun eine Sprache $L \subseteq T_\Sigma$ von B\"aumen. Wir definieren 
$\equiv_L$ als:\\
$u \equiv_L v \Leftrightarrow \forall C \in T_\Sigma (x): C[u] \in L \Leftrightarrow 
C[v] \in L$.\\

Beispiel: Betrachte alle B\"aume der Form \Tree [.f [.f [.... [.f a b ] ] b ] b ]

$L = \{ f(f(\dots f(a, b), b) \dots , b) \}$ \\

Dann gilt:  \Tree [.f a b ] $\equiv$ \Tree [.f [.f a b ] b ] \\

$C = $ \Tree [.f x b ],
$C' = $ \Tree [.f [.f x b ] b ],
$C'' = $ \Tree [.f a x ]

\subsection{Theorem (Myhill-Nerode)}

Die folgenden Aussagen sind \"aquivalent:\\

a) L ist erkennbar

b) L ist die Vereinigung von \"Aquivalenzklassen einer Kongruenz mit endlichem Index

c) $\equiv_L$ hat endlichen Index \\

Beweis:\\
\glqq a $\Rightarrow$ b \grqq:
Sei $\mathcal{A}$ vollst\"andiger DTA mit $L(\mathcal{A}) = L$.
Sei $\mathcal{A} = (Q, \Sigma, F, \Delta)$.\\
Definiere $u \equiv_\mathcal{A} v \Leftrightarrow \delta (u) = \delta (v)$.

Offensichtlich hat $\equiv_\mathcal{A}$ h\"ochstens $|Q|$-viele \"Aquivalenzklassen.
Au\ss erdem ist $\equiv_\mathcal{A}$ eine Kongruenz.
Nun ist $L$ Vereinigung aller Klassen $[u]_{\equiv_\mathcal{A}}$ mit $\delta(u) \in F$.\\

\glqq b $\Rightarrow$ c \grqq:
Sei $\sim $ eine Kongruenz mit dnlichem Index. Sei $u \sim  v$.
Wegen Lemma 1.20 gilt \\
$C[u] \sim  C[v] \forall C \in T_\Sigma (x)$.
Nun ist aer $L$ die Vereinigung von \"Aquivalenzklassen von $\sim $, das hei\ss t 
$C[u] \in L \Leftrightarrow C[v] \in L$.
Insbesondere ist also $u \equiv_L v$\\
Wir haben gezeigt: $v \in [u]_\sim  \Rightarrow v \in [u]_{\equiv_L}$, also 
$[u]_\sim  \leq [u]_{\equiv_L}$\\
(Also ist $\sim $ eine Verfeinerung von $\equiv_L$)\\
Insbesondere hat $\equiv_L$ kleinern Index als $\sim $, also endlichen.\\

\glqq c $\Rightarrow$ a \grqq:
Die Zust\"ande $Q_\text{min}$ sind die \"Aquivalenzklassen bez\"uglich $\equiv_L$.
(Damit ist $Q_\text{min}$ endlich). Wir definieren Regeln\\
$f([u_1], \dots, [u_n]) \rightarrow [f(u_1, \dots, u_n)]$.\\
Das ist wohldefiniert, weil $\equiv_L$ eine Kompetenz ist.
Finalzust\"ande $F_\text{min}$ sind $\{ [u]_{\equiv_L} | u \in L \}$.\\
Dann akzeptiert $\mathcal{A}_\text{min} = (Q_\text{min}, \Sigma, F_\text{min}, 
\Delta_\text{min})$ die Sprache $L$.\\

Beispeiel:\\
Betrachte $\Sigma = \{a^{(0)}, g^{(1)}, f^{(2)} \}$ und \\
$L = \{ f(g^i(a), g^i(a)) | i \geq 0\}$\\

Betrachte $g^i(a)$ und $g^j(a)$ mit $i \neq j$.
Dann ist $C^i = f(x, g^i(a))$ ein Kontext mit $C^i[g^i(a)] \in L$, aber 
$C^i[g^j(a)] \notin L$.
Da es unenlich viele $g^i(a)$ gibt, hat die Kongruenz bez\"uglich $L$ unendlichen Index, 
also ist $L$ nicht erkennbar.

\subsection{Korollar}

Ist $L$ erkennbar, gibt es einen bis auf Umbenennung der Zust\"ande eindeutigen, 
vollst\"andigen DBA $\mathcal{A}$ mit $L = L(\mathcal{A})$.
Dieser ist $\mathcal{A}_\text{min}$ aus obigem Beweis.\\
Beweis:\\
Sei $L = L(\mathcal{A})$. Vorher gesehen:\\
$\equiv_\mathcal{A}$ ist Verfeinerung von $\equiv_L$\\
Also ist $|Q| \geq |Q_\text{min}|$. Wir nehmen OBDA an:
beide reduziert. Sei nun $q in Q$. Getrachte ein $u \in T_\Sigma$ mit $\delta(u) = q$.
Betrachte die Funktion $\rho : Q \to Q_\text{min}$ mit 
$\delta(u) = q \mapsto \delta_\text{min}(u)$\\
Die Abbildung $\rho$ ist wohldefiniert, denn falls $\delta (u) = \delta (v)$, dann 
$u \equiv_\mathcal{A} v \Rightarrow u \equiv_L v \Leftrightarrow \delta_\text{min} (u) = 
\delta_\text{min} (v)$.
Au\ss erdem ist $\rho$ surjektiv, denn $\delta_\text{min} (u)$ hat das Urbild $\delta(u)$.\\
Also: $|Q| = |Q_\text{min}| \Rightarrow \rho$ ist Bijektion. $\Box$

\subsection{Einschub - Homomorphismen von Baumsprachen}

\subsubsection{Allgemeine Homomorphismen}

$(M, \cdot), (N, \ast); h: M \to N$ hei\ss t Homomorphismus, falls \\
$\forall m, \hat{m}: h(m \cdot \hat{m}) = h(m) \ast h(\hat{m})$

\subsubsection{Worthomomorphismen}

$(A^\ast, \cdot), (B^\ast, \cdot); h: A^\ast \to B^\ast$ ist Homomorphismus, falls
$h(w \cdot \hat{w}) = h(w) \cdot h(\hat{w})$.
(zus\"atzlich $h(\varepsilon) = \varepsilon$)\\

Nutze f\"ur die Definition des Homomorphismus die induktive Definition von W\"ortern aus 
$A^\ast$.

0.) $\varepsilon \in A^\ast$\\
1.) $a \in A^\ast$ $\forall a \in A$\\
2.) $a \cdot w \in A^\ast$ $\forall a \in A, w \in A^\ast$\\

Ein Wort-Homomorphismus entsteht deshalb aus einer Abbildung $\bar{h}: A \to B$
wie folgt:

0.) $h(\varepsilon) = \varepsilon$\\
1.) $h(a) = \bar{h}(a)$ $\forall a \in A$\\
2.) $h(a \cdot w) = h(a) \cdot h(w) = \bar{h}(a) \cdot h(w)$ $\forall a \in A, w \in A^\ast$\\

\subsubsection{Baumhomomorphismen}

$T_\Sigma:$\\
1.) $a \in T_\Sigma$ $\forall a^{(0)} \in \Sigma$\\
2.) $f(t_1, \dots, t_n) \in T_\Sigma$ $\forall f^{(n)} \in T_\Sigma$\\

Zun\"achst: Schreibe $\Sigma = \bigcup\limits_{n=0}^{r} \Sigma^{(n)}$, wobei
$\Sigma^{(n)} = rk^{-1}(n)$.

Sei $X = \{x_1, \dots, x_n\}$ und $X_n = \{x_1, \dots, x_n\}, X_0 = \emptyset$\\

Dann ist f\"ur ein Rangalphabet $\Gamma$ auch $\Gamma \cup X_n$ ein Rangalphabet mit 
$rk(x_i) = 0$\\

Nachtrag - Substitution:\\
$t, s \in T_{\Sigma \cup X_n}$ (f\"ur ein $n \in \mathbb{N}$)\\
Sei $P \subseteq pos(t), P = \{ p \in pos(t) | t(p) = x_i\}$ f\"ur ein $i \in \{1, \dots, n\}$\\
Etwa $P = \{p_1, \dots, p_m\}$\\
Dann ist $t_{[x_i \leftarrow s]} = t[s]_{p_1} \cdot \dots \cdot t[s]_{p_m}$.\\

Definition von Homomorphismen f\"ur jeden Rang $n = 0, \dots, r$\\
W\"ahle eine Funktion $\bar{h}_n: \Sigma^{(n)} \to T_{\Gamma \cup X_n}$\\
Der von $\bar{h}_0, \dots, \bar{h}_r$ erzeugte Homomorphismus $h: T_\Sigma \to T_\Gamma$\\

1.) $h(a) = \bar{h}_0(a)$ $\forall a \in \Sigma$\\
2.) $h( f(t_1, \dots, t_n)) = \bar{h}_n (f) [x_1 \leftarrow h(t_1)] \dots [x_n \leftarrow h(t_n)]$\\

Beispiel:\\

$\Sigma = \{ a^{(0)}, g^{(1)}, f^{(2)} \}$\\
$\Gamma = \{ \alpha^{(0)}, \delta^{(2)}, \tau^{(2)} \}$\\

$\bar{h}_0: a \rightarrow $ \Tree [.$\tau$ $\alpha$ $\alpha$ ] ;
$\bar{h}_1: g \rightarrow $ \Tree [.$\rho$ [.$\rho$ $\alpha$ $x_1$ ] $x_1$ ] ;
$\bar{h}_2: f \rightarrow $ \Tree [.$\tau$ [.$\tau$ $x_2$ $\alpha$ ] $x_2$ ] \\

Gegenbeispiel:\\

$\bar{h}_0: a \rightarrow $ \Tree [.$\tau$ $\alpha$ $x_1$ ] ;
$\bar{h}_1: g \rightarrow $ \Tree [.$\rho$ [.$\rho$ $\alpha$ $x_2$ ] $x_1$ ] ;
$\bar{h}_2: f \rightarrow $ \Tree [.$\tau$ $x_5$ $\alpha$ ] \\

Erzeugter Homomorphismus $h: T_\Sigma \to T_\Gamma$ wie oben\\

$h($ \Tree [.f a [.g a ] ] $)$\\
$h(a) = \bar{h}_0(a) = $ \Tree [.$\tau$ $\alpha$ $\alpha$ ]\\
$h(g(a)) = \bar{h}_1(g)[x_1 \leftarrow h(a)] = $ 
\Tree [.$\rho$ [.$\rho$ $\alpha$ [.$\tau$ $\alpha$ $\alpha$ ] ] [.$\tau$ $\alpha$ $\alpha$ ] ]\\
$h(f(a, g(a))) = \bar{h}_2(f)[x_1 \leftarrow h(a)][x_2 \leftarrow h(g(a))]$
\Tree [.$\tau$ 
	[.$\tau$ 
		[.$\rho$ [.$\rho$ $\alpha$ [.$\tau$ $\alpha$ $\alpha$ ] ] [.$\tau$ $\alpha$ $\alpha$ ] ] 
		$\alpha$ ] 
	[.$\rho$ 
		[.$\rho$ $\alpha$ [.$\tau$ $\alpha$ $\alpha$ ] ] 
		[.$\tau$ $\alpha$ $\alpha$ ] ] ]\\

\subsubsection{lineare Terme}

Ein Term $t \in T_{\Sigma \cup X_n}$ hei\ss t linear, falls jede Variable h\"ochstens einmal 
vorkommt, \\
d.h. falls $\forall i \in \{1, \dots , n\}: | \{ p \in pos (t) | t(p) = x_i \} | \leq 1$

\subsubsection{linearer Homomorphismus}

Ein Homomorphismen $h: T_\Sigma \to T_\Gamma$ erzeugt von $\bar{h}_0 , \dots, \bar{h}_r$ 
hei\ss t linear, falls $\forall f^{(n)} \in T_\Sigma$ gilt:
$\bar{h}_n(f) \in T_{\Gamma \cup X_n}$ ist linear.\\

Beispiel: im Allgemeinen erhalten Homomorphismen die Erkennbarkeit nicht.\\
Betrachte\\ 
	$\Sigma = \{ f^{(1)}, g^{(1)}, a^{(0)} \}, 
	\Gamma = \{ \hat{f}^{(2)}, g^{(1)}, a^{(0)} \}\\
	\bar{h}_0: a \to a, \bar{h}_1: g \to g(x_1), \bar{h}_1: f \to \hat{f}(x_1, x_1)$

F\"ur $L \subseteq T_\Sigma, L = \{f(g^i(a)) | i \in \mathbb{N} \}$ ist
$h(L) = \{\hat{f}(g^i(a), g^i(a)) | i \in \mathbb{N} \}$ nicht erkennbar, 
obwohl $L$ erkennbar ist.

\subsubsection{Satz}

Sei $L \subseteq T_\Sigma$ erkennbar, $h: T_\Sigma \to T_\Gamma$ ein linearer Homomorphismen,
dann ist $h(L) \subseteq T_\Gamma$ erkennbar.\\

Beweisskizze:\\
Sei $A = (Q, \Sigma, F, \Delta)$ ein reduzierter DFTA mit $L(A) = L$

\begin{itemize}
	\item Seien $(\bar{h}_n)_{n=0}^r$ die erzeugenden Funktionen von $h$
	\item Definiere NFTA $A' = (Q', \Gamma, F', \Delta)$ wie folgt:\\
		f\"ur jede Regel $\rho: f(q_1, \dots , q_n) \to q \in \Delta:$\\
			Setze $Q^\rho = \{ q_p^\rho | q \in pos(\bar{h}_n(f))\}$ und\\
			$\Delta ^\rho$ f\"ur jedes $p \in pos(\bar{h}_n(f))$\\
			falls $(\bar{h}_n f)(p) = g^{(k)} \in \Gamma^{(k)}$ f\"ur ein $k$,\\
			$g(q_{p \cdot 1}^\rho, \dots, q_{p \cdot k}^\rho) \to q_p \in \Delta^\rho$\\
			falls $(\bar{h}_n f)(g) = x_i$ f\"ur ein $i$\\
			$q_i \to q_p^\rho \in \Delta^\rho$\\
			$q_\varepsilon^\rho \to q \in \Delta^\rho$\\
			OBdA $Q^\rho$ paarweise disjukt auch mit $Q$\\
	\item $Q' = \bigcup\limits_{\rho \in \Delta} Q^\rho \cup Q$
	\item $\Delta = \bigcup\limits_{\rho \in \Delta} \Delta^\rho \cup Q$
	\item $F' = F$ 
\end{itemize}

\subsubsection{Satz}

Sei $h: T_\Sigma \to T_\Gamma$ beliebiger Homomorphismus und $L \subseteq T_\Gamma$ erkennbar.
Dann ist auch $h^{-1}(L) \subseteq T_\Sigma$ erkennbar.\\

Beweis:\\
Sei $A' = (Q', \Gamma, F', \Delta')$ ein vollst\"andiger DFTA mit $L(A') = L$.\\
Definiere $A = (Q, \Sigma, F, \Delta)$ wie folgt:\\
$Q = Q', F = F'$\\
$f(q_1, \dots, q_n) \to q \in \Delta \\
\Leftrightarrow \bar{h}_n (f)[x_1 \leftarrow q_1] \cdot \dots \cdot 
[x_n \leftarrow q_n] \to_{\mathcal{A}'}^\ast q$\\

Beweis der Korrektheit (strukturelle Induktion):\\
Zeige die st\"arkere Aussage: f\"ur $q \in Q = Q'$ gilt \\
$t \to_\mathcal{A}^\ast q \Leftrightarrow h(t) \to_{\mathcal{A}'}^\ast q$\\
($h^{-1}(s) \to_\mathcal{A}^\ast q \Leftrightarrow s \to_{\mathcal{A}'}^\ast q$)\\

Induktionsannahme:\\
Sei $t = a \in \Sigma^{(0)}$. Dann gilt
$h(a) \to_{\mathcal{A}'}^\ast q \Leftrightarrow
\bar{h}_0(a) \to_{\mathcal{A}'}^\ast q \Leftrightarrow
a \to q \in \Delta \Leftrightarrow a \to_\mathcal{A}^\ast q$\\

Induktionsvoraussetzung:\\
Die Aussage gelte f\"ur Terme mit H\"ohe $\leq k$\\

Induktionsschritt: Dann gilt sie auf f\"ur $t$ mit H\"ohe $k+1$.\\
Sei $t = f(t_1, \dots, t_n)$ mit $ht(t_i) \leq k$ $\forall i \in \{1, \dots, m\}$\\

$t \to_\mathcal{A}^\ast q \Leftrightarrow \exists q_1, \dots, q_n: t_i \to_\mathcal{A}^\ast q_1$
und $f(q_1, \dots, q_n) \to q \in \Delta$\\
$\Leftrightarrow \exists q_1, \dots, q_n: h(t_i) \to_{\mathcal{A}'}^\ast q_i$ und
$f(q_1, \dots, q_n) \to q \in \Delta$ \\
$\Leftrightarrow \exists q_1, \dots, q_n: h(t_i) \to_{\mathcal{A}'}^\ast q_i$ und
$\bar{h}_n(f)[x \leftarrow q_i] \cdot \dots \cdot [x_n \leftarrow q_n] \to_{\mathcal{A}'}^\ast q$\\
$\Leftrightarrow \bar{h}_n(f)[x_1 \leftarrow h(t_1)] \cdot \dots \cdot 
[x_n \leftarrow h(t_n)] \to_{\mathcal{A}'}^\ast q$\\
$\Leftrightarrow h(f(t_1, \dots, t_n)) (=h(t)) \to_{\mathcal{A}'}^\ast q$

\subsection{Top-Down Baumautomaten}

Bisher: Bottom-Up TA - laufen B\"aume von den Bl\"attern zu der Wurzel nach oben.\\
Nun: Top-Down TA - umgekehrt\\

Definition:\\
Ein nicht-deterministischer Top-Down Baumautomat ist ein Tupel 
$\mathcal{A} = (Q, \Sigma, I, \Delta)$ , wobei:

\begin{itemize}
	\item $Q$ eine enliche Menge Zust\"ande
	\item $\Sigma$ ein Rangalphabet
	\item $I \subseteq Q$ Initialzust\"ande
	\item $\Delta$ eine endliche Menge Regeln der Form\\
	$q(f(x_1, \dots, x_n)) \to f(q_1(x_1), \dots, q_n(x_n))$\\
	bzw. f\"ur $n = 0$: $q(a) \to a$
\end{itemize}
ist.\\

Beispiel:\\
Wir betrachten wieder $\Sigma = \{a^{(0)}, b^{(0)}, f^{(2)}\}$ und $t =$
\Tree [.f [.f a b ] b ]

Setze \\
$Q = \{ q_f(f(x_1, x_2)) \to f(q_f(x_1),q_b(x_2)),
        q_f(f(x_1, x_2)) \to f(q_a(x_1),q_b(x_2)),
        q_a(a) \to a, q_b(b) \to b\}$\\

Run f\"ur $t$ (intuitiv):\\

\Tree [.$q_f$ [.f [.f a b ] b ] ] $\to$
\Tree [.f [.$q_f$ [.f a b ] [.$q_b$ b ] ] ] $\to$
\Tree [.f [.f [.$q_a$ a ] [.$q_b$ b ] ] b ] $\to$
\Tree [.f [.f a b ] b ]\\

Betrachte folgende \"Ubergangsrelation:\\
$C[q(f(t_1, \dots, t_n))] \to C[f(q_1(t_1), \dots, q_n(t_n))]$, $C \in T_{\Sigma \cup Q} (x)$,\\
falls $q(f(x_1, \dots, x_n)) \to f(q_1(x_1), \dots, q_n(x_n)) \in \Delta$\\

$\to^\ast$ transitive H\"ulle\\

$L(\mathcal{A}) = \{t \in T_\Sigma | q(t) \to^\ast, q \in I\}$\\

\subsection{Satz}

Eine Sprache $L \subseteq T_\Sigma$ ist genau dann erkennbar, wenn es einen nichtdeterministischen
Top-Down TA $\mathcal{A}$ gibt mit $L = L(\mathcal{A})$.\\

Bemerkung: \\
Top-Down Automaten sind nicht determinisierbar. Deterministische Top-Down TA erkennen 
\glqq path closed\grqq \ Sprachen.\\

\section{Grammatiken}

\subsection{Definition - Grammatik}

Eine Grammatik ist ein Tupel $G = (S, N, \Sigma, R)$. Dabei ist:

\begin{itemize}
	\item $S$ Startsymbol ($S = S^{(0)}, S \in N$)
	\item $N$ (Rangalphabet) nichtterminale Symbole
	\item $\Sigma$ (Rangalphabet) terminale Symbole ($\Sigma \cap N = \emptyset$)
	\item $R$ Regeln der Form $\alpha \to \beta$ mit:\\
		$\alpha, \beta \in T_{\Sigma \cup N \cup X}$ ($X \cap (\Sigma \cup N) = \emptyset$ Variablen),\\
		$\alpha$ enth\"alt mindestens ein Nichtterminal-Symbol
\end{itemize}

Eine regul\"are Grammatik enth\"alt nur Regelnd der Form
$A \to B$, wobei $A$ den Rang $0$ hat und $b \in T_{\Sigma \cup N}$.
Insbesondere enth\"alt $N$ nur Symbole mit Rang $0$.\\

Beispiel: Betrachte $G = (S, \{S, A\},\{a^{(0)}, b^{(0)}, f^{(2)}\},R)$ mit\\
$R = \{S \to f(A, b), A \to a, A \to f(A, b)\}$\\
zum Beispiel haben wir:\\

$S$ $\to$
\Tree [.f A b ] $\to$
\Tree [.f [.f A b ] b ] $\to$
\Tree [.f [.f a b ] b ]

Beispiel: Betrachte $G = (S, \{S, A\},\{a^{(0)}, g^{(3)}, f^{(2)}\},R)$ mit\\
$R = \{S \to A, A \to f(a, b), A \to g(a, A, b)\}$\\
zum Beispiel:\\

$S$ $\to$
$A$ $\to$
\Tree [.g a A b ] $\to$
\Tree [.g a [.g a A b ] b ] $\to$
\Tree [.g a [.g a [.f a b ] b ] b ]\\

Bemerkung - kontextfreie Baumgrammatik:\\
$F(x_1, \dots, x_n) \to t$, $t \in T_{\Sigma \cup N \cup \{x_1, \dots, x_n\}}$\\
z.B. $S \to F(a,a), F(x,x) \to F(G(x),G(x)), F(x,x) \to f(x,x), G(x) \to g(x)$\\

Damit k\"onnen wir erzeugen:\\
$S \to F(a,a) \to F(G(a),G(a)) \to F(G(G(a)),G(G(a))) \to \dots \to f(g(g(a)),g(g(a)))$\\
\Tree [.f [.g [.g a ] ] [.g [.g a ] ] ]\\

Betrachte nun folgende Ableitungsrelation f\"ur regul\"are Grammatiken:\\
Wir schreiben $s \to_G$ genau dann wenn ein Kontext $C \in T_{\Sigma \cup N}(x)$ existiert,
sodass\\ $s = C[A], t = C[\alpha], A \to \alpha \in R$ f\"ur $G = (S, N, \Sigma, R)$.\\
Mit $\to_G^\ast$ bezeichnen wir die transitive H\"ulle von $\to_G$.

\subsection{Definition}

Ist $G$ eine regul\"are Grammatik, dann hei\ss t $L(G) = \{t \in T_\Sigma | S \to_G^\ast t\}$
die von $G$ akzeptierte Sprache. Eine Sprache $L \subseteq T_\Sigma$ hei\ss t regul\"ar, falls
$L = L(G)$ f\"ur eine regul\"are Grammatik.

Betrachte regul\"are Grammatik $G = (S,N,\Sigma,R)$
Wir bezeichnen mit $L_G(A)$, $A \in N$, die von $G$ erzeugte Sprache mit $A$ als Startsymbol.

\subsection{Definition - reduziert}

Sei $G = (S,N,\Sigma,R)$ eine regul\"are Grammatik und $ A \in N $.
Dann hei\ss t $A$
\begin{itemize}
	\item erreichbar, falls ein Kontext $C \in T_\Sigma(x)$ existiert, so dass
		$S \to_G^\ast C[A]$
	\item produktiv, falls $L_G(A) \neq \emptyset$
\end{itemize}

G hei\ss t reduziert, falls alle $A \in N$ erreichbar und produktiv sind.

\subsection{Satz}

Ist $G$ eine regul\"are Grammatik mit $L(G) = L$, dann existiert eine reduzierte regul\"are 
Grammatik $G'$ mit $L(G') = L$.\\

Nun: zu $G$ eine \glqq Normalform \grqq konstruieren.

\subsection{Definition - Normalisierung}

Eine regul\"are Grammatik $G$ hei\ss t normalisiert, falls alle Regeln aus $R$ die Form
\begin{itemize}
	\item $A \to f(A_1, \dots, A_n)$, $A, A_1, \dots, A_n \in N$, $v \in \Sigma^{(n)}$
	\item $A \to a$, $A \in N$, $a \in \Sigma$
\end{itemize}

\subsection{Satz}

Ist $G$ eine regul\"are Grammatik mit $L(G) = L$, dann existiert eine normalisierte regul\"are 
Grammatik $G'$ mit $L(G') = L$.\\

Beweis: Wir ersetzen Regeln der Form 
$A \to f(s_1, \dots, s_n)$ durch
$A \to f(A_1, \dots, A_n)$ wobei:
Ist $s_i \in N$, dann ist $A_i= s_i$, ansonsten ist $A_i$ ein neues Symbol und wir f\"ugen 
$A_i \to s_i$ hinzu.\\
...Iterieren...\\

Es bleiben \"ubrig:\\
$A \to f(A_1, \dots, A_n)$, $A \to a \in \Sigma$, $A_i \to A_j$ (letztere \"uberbr\"ucken)\\

Wir erhalten nun:

\subsection{Theorem}

$L$ ist erkennbar $\Leftrightarrow$ $L$ ist regul\"ar.\\

Beweis:\\
\glqq $\Rightarrow$ \grqq
Ist $L$ erkennbar, so existiert ein nicht-deterministischer Top-Down-TA 
$\mathcal{A} = (Q, \Sigma, I, \Delta)$ mit $L = L(\mathcal{A})$.
Betrachte eine Grammatik $G = (S, N, \Sigma, R)$ mit
\begin{itemize}
	\item $S$ ist ein neues Symbol
	\item $N = \{ A_q | q \in Q\}$
	\item $R = \{ A_q \to f(A_{q_1}, \dots, A_{q_n}) | 
		q(f(x_1, \dots, x_n)) \to f(q_1(x_1)), \dots, q_n(x_n) \in \Delta \} \cup 
		\{ S \to A_{q_i} | q_i \in I \}$
\end{itemize}

Offensichtlich ist $L(G) = L(A)$

\glqq $\Leftarrow$ \grqq analog (gehe von normalisierter Grammatik aus) $\Box$\\

Ziel: Definieren Konkatenation und Kleene-Stern

Problem hierbei ist: Wir m\"ussen erkl\"aren, wie wir B\"aume zusammensetzen.

Dazu definieren wir: Substitution von Sprachen.\\

Betrachte $t \in T_{\Sigma \cup k}$, wobei $k = \{\Box_1, \dots, \Box_n\}$, 
$\Box_i$ sind Konstanten.\\
Es seien $L_i \subseteq T_{\Sigma \cup k}$ Sprachen. Dann ist die Substitution
$t\{ \Box_1 \leftarrow L_1, \dots, \Box_n \leftarrow L_n \}$ induktiv wie folgt definiert:
\begin{itemize}
	\item $\Box_i\{ \Box_1 \leftarrow L_1, \dots, \Box_n \leftarrow L_n \} = L_i$
	\item $a\{ \Box_1 \leftarrow L_1, \dots, \Box_n \leftarrow L_n \} = \{a\}$, $a \neq \Box_i$, $a \in \Sigma^0$
	\item $f(s_1, \dots, s_n)\{ \Box_1 \leftarrow L_1, \dots, \Box_n \leftarrow L_n \} = 
		\{f(t_1, \dots, t_n) | t_i \in s_i\{ \Box_1 \leftarrow L_1, \dots, \Box_n \leftarrow L_n \} \}$
\end{itemize}

Au\ss erdem setzen $L\{ \Box_1 \leftarrow L_1, \dots, \Box_n \leftarrow L_n \} = 
\bigcup\limits_{t \in \L} t\{ \Box_1 \leftarrow L_1, \dots, \Box_n \leftarrow L_n \}$

Beispiel: Betrachte $k = \{\Box_1, \Box_2\}$\\
$t =$ \Tree [.f [.f $\Box_1$ $\Box_2$ ] $\Box_2$ ]\\
$L_2 = \{a, b\}$

Dann ist $t\{ \Box_2 \leftarrow L_2 \} = \{$
\Tree [.f [.f $\Box_1$ a ] a ],
\Tree [.f [.f $\Box_1$ a ] b ],
\Tree [.f [.f $\Box_1$ b ] a ],
\Tree [.f [.f $\Box_1$ b ] b ] $\}$\\

Nun setzen wir f\"ur zwei Sprachen $L, M$:\\
$L \cdot_\Box M = \bigcup\limits_{t \in L} t \{ \Box \leftarrow M\}$

Des weiteren ergibt sich der Kleene-Stern:
\begin{itemize}
	\item $L^{0.\Box} = \{ \Box \}$
	\item $L^{n+1.\Box} = L^{n.\Box} \cup L \cdot_\Box L^{n.\Box}$
\end{itemize}

$\Rightarrow L^{\ast.\Box} = \bigcup\limits_{n \geq L^{n.\Box}}$

Beispiel: Betrachte $L = \{$\Tree [.f $\Box$ $\Box$ ]$, a\}$\\
$L^{0.\Box} = \{\Box\}$\\
$L^{1.\Box} = \{\Box\} \cup L \cdot_\Box \{ \Box \} = 
\{\Box\} \cup \{$\Tree [.f $\Box$ $\Box$ ] $,a\}$\\
$L^{2.\Box} = \{\Box, $\Tree [.f $\Box$ $\Box$ ] $,a\} \cup
L \cdot_\Box \{\Box, $\Tree [.f $\Box$ $\Box$ ] $,a\} = \dots$

Abschlusseingenschaften:

\subsection{Satz}

Es sei $L \subseteq T_{\Sigma \cup k}$ regul\"ar, sowie $\Box_1, \dots, \Box_n \in k$.
Dann ist $L\{ \Box_1 \leftarrow L_1, \dots, \Box_n \leftarrow L_n \}$ regul\"ar.\\

Beweis:\\
Betrachte normalisierte Grammatiken $G, G_1, \dots, G_n$ mit $L(G) = L, L(G_1) = L_1, \dots, L(G_n) = L_n$,
$G = (S,N,\Sigma \cup k,R), G_i =  (S_i,N_i,\Sigma \cup k, R_i)$. (alle Nichtterminale paarweise disjunkt)\\
Konstruiere $G' = (S,N',\Sigma \cup k,R')$ mit:
\begin{itemize}
	\item $N' = N \cup N_1, \cup \dots \cup N_n$
	\item $R'$ enth\"alt alle Regeln in $R, R_1, \dots , R_n$, wobei $A \to \Box_i$ ersetzt werden durch
		$A \to S_i$
\end{itemize}

Direkt zeigen: $L\{ \Box_1 \leftarrow L_1, \dots, \Box_n \leftarrow L_n \} \subseteq L(G')$\\
Wir zeigen \glqq $\supseteq$ \grqq\\
Induktion: \"uber die Anzahl der Ableitungsschritte zeigen wir
$A \to_{G'}^\ast s'$ mit $S \in T_{\Sigma \cup k}$ | s' enth\"alt keine Nichtterminale\\
$\exists s$ mit $A_{G'}^\ast$ und $s' \in s\{ \Box_1 \leftarrow L_1, \dots, \Box_n \leftarrow L_n \}$ |
das hei\ss t $s" \in L_G(A)\{ \Box_1 \leftarrow L_1, \dots, \Box_n \leftarrow L_n \}$\\

Induktionsannahme: Angenommen $A \to_G^\ast$ in einem Schritt, das hei\ss t $s' \in L$, die Regel
kann nicht $A \to \Box_i$ sein (existiert nicht in $G'$) und auch nicht $A \to s_i$ 
(kein $s' \in T_{\Sigma \cup k}$. Damit: $S' \in L$. Seze $s = s'$, damit enth\"alt $s$ kein $\Box_i$, also
$\{s'\} = \{s\} = s\{ \Box_1 \leftarrow L_1, \dots, \Box_n \leftarrow L_n \}$, also insbesondere
$s' \in s\{ \Box_1 \leftarrow L_1, \dots, \Box_n \leftarrow L_n \} \subseteq 
L(A)\{ \Box_1 \leftarrow L_1, \dots, \Box_n \leftarrow L_n \}$\\
Induktionsschritt: $A \to_{G'}^\ast s'$: zerlege $A \to){G'} s_1 \to^\ast_{G'} s'$
F\"alle f\"ur $s_1$ (bzw. $A \to s_1$ )
\begin{itemize}
	\item $a \to f(A_1, \dots, A_m) \Rightarrow s' = f(t_1, \dots, t_n)$. Laut Induktionsvoraussetzung:
		$t_i \in L(A_i)\{ \Box_1 \leftarrow L_1, \dots, \Box_n \leftarrow L_n \} \Rightarrow s' \in L(A)$
	\item $A \to s_i \in R' \Rightarrow A \to \Box_i \in R$ (laut Konstruktion) 
		$\Rightarrow \Box_i \in L(A) \Rightarrow s' \in L_i \Rightarrow s' \in 
		L(A)\{ \Box_1 \leftarrow L_1, \dots, \Box_n \leftarrow L_n \}$
\end{itemize}

Aussage gilt f\"ur  alle Nichtterminale $A_i$, also auch f\"ur Startsymbol $S$ . $\Box$

\subsection{Satz}

Ist $L \subseteq T_{\Sigma \cup K}$ regul\"ar und $\Box \in K$, dann ist $L^{\ast . \Box}$ regul\"ar.\\
Beweis: Betrachte normalisierte Grammatik $G = (S, N, \Sigma \cup K, R)$ mit $L(G) = L$.\\
Konstriere $G' = (S', N \cup \{S'\}, \Sigma \cup K, R')$ (mit $S' \notin N$) wie folgt:\\
$R'$ enth\"alte alle Regeln aus $R$, wobei:
\begin{itemize}
	\item $A \to \Box$ wird ersetzt durch $A \to S'$
	\item $S' \to S$
	\item $S' \to \Box$ (damit $\Box \in L(G')$)
\end{itemize}

Induktion liefert: $L(G') = L^{\ast . \Box}$

\subsection{Definition}

Wir formalisieren nun die rationalen Ausdr\"ucke:\\
Sei $\Sigma$ ein Rangalphabet, $K$ eine Menge Konstanten mit $\Box \in K$:
Dann ist $Rat(\Sigma, K)$ die kleinste Menge, sodass:
\begin{itemize}
	\item $\emptyset \in Rat(\Sigma, K)$
	\item $a \in \Sigma^0 \cup K 'Rightarrow a \in Rat(\Sigma, K)$
	\item $f \in \sigma^n, E_1, \dots, E_n \in Rat(\Sigma, K) \Rightarrow f(E_1, \dots, E_n) \in Rat(\Sigma, K)$
	\item $E_1, E_2 \in Rat(\Sigma, K) \Rightarrow E_1 \cup E_2 \in Rat(\Sigma, K)$
	\item $E_1, E_2 \in Rat(\Sigma, K), \Box \in K \Rightarrow E_1 \cdot _\Box E_2 \in Rat(\Sigma, K)$
	\item $E_1 \in Rat(\Sigma, K), \Box \in K \Rightarrow E_1^{\ast . \Box} \in Rat(\Sigma, K)$
\end{itemize}

Die Ausdr\"ucke $Rat(\Sigma, K)$ hei\ss en rational \"uber $\Sigma$ und $K$.
Ist $E$ ein ratoinaler Ausdruck in $Rat(\Sigma, K)$, dann repr\"asentiert $E$ eine Menge von Termen aus 
$T_{\Sigma \cup K}$, bezeichnet mit $||E||$.

\begin{itemize}
	\item $||\emptyset|| = \emptyset$
	\item $||a|| = \{a\}$
	\item $f(E_1, \dots, E_n)|| = \{f(t_1, \dots, t_n) | t_i \in ||E_i||\}$
	\item $||E_1 \cup E_2|| = ||E_1|| \cup ||E_2||$
	\item $||E_1 \cdot E_2|| = ||E_1||\{\Box \Leftarrow ||E_2||\}$
	\item $||E^{\ast . \Box}|| = ||E||^{\ast . \Box}$
\end{itemize}

Beispiel: Der Ausdruck $(f(\Box, b))^{\ast . \Box} \cdot_\Box a$ ist rational und repr\"asentiert 
alle Terme der Form

\Tree [.f a b ],
\Tree [.f [.f a b ] b ],
\Tree [.f [.f [.f a b ] b ] b ], ...

Wir erhalten die \glqq Baumautomaten \grqq - Variante von Kleene:\\

\subsection{Theorem}

Rationale Ausdr\"ucke haben die selbe Ausdrucksst\"arke wie bottom-up Baumautomaten.\\
Beweis: Ist $E$ ein rationaler Ausdruck, so existiert laut S\"atzen 2.8 und 2.9 und den 
Abschlusseigenschaften von Baumautomaten einen Baumautomaten $\mathcal{A}$ mit $L(\mathcal{A}) = ||E||$.
Umgekehrt: Sei $\mathcal{A} = (Q, \Sigma, F, \Delta)$ ein Baumautomat.
Wir zeigen: Es existiert ein rationaler Ausdruck $E$ aus $Rat(\Sigma. K)$ mit $||E|| = L(\mathcal{A})$.

Betrachter hierzu f\"ur $1 \leq i,j \leq |Q|$ und $Z \subseteq Q$ Terme $T(i, j, Z)$ definiert wie folgt:
$t \in T(i, j, Z)$, falls $t \in T_{\Sigma \cup K}$ mit einem Lauf $R$ in $t$, so dass:
\begin{itemize}
	\item $v(\varepsilon) = q_i$
	\item ist $p \neq \varepsilon$ und $t(p) /in Z$, dann ist $r(p) \in \{q_1, \dots, q_j\}$
\end{itemize}

Damit: $L(\mathcal{A}) = \bigcup\limits_{q_i \in F} T(i, |Q|, \emptyset)$

Induktion \"uber $j$: $T(i, j, Z) \in Rat(\Sigma, K)$
$j = 0$: $t \in T(i, 0, Z)$ bedeutet, es existiert ein Lauf $r$
\begin{itemize}
	\item $r(\varepsilon) = q_i$
	\item kein Symbol, das nicht nullstellig ist, darf gelabelt sein
		$\Rightarrow t = a$ oder $t = f(a_1, \dots, a_n), a_1, \dots, a_n \in \Sigma^0$
		$\Rightarrow$ endlich viele, also existiert ein rationaler Ausdruck
\end{itemize}

Induktionsschritt: Angenommen f\"ur $j' < j$ ist $T(i, j', Z) \in Rat(\Sigma, Q)$.\\
Schreibe $T(i, j, Z) = T(i, j-1, Z) \cup T(i, j-1, Z \cup \{q_j\}) \cdot_{q_j} (T(j, j-1, Z \cup \{q_j\}))^{\ast . q_j} \cdot_{q_j} T(j, j-1, Z)$\\

Ziel: Zusammenhang zwischen Baum- und Wortsprachen\\
Neben der Pfadsprache (siehe \"Ubung) betrachten wir den sogenannten Yield-Operator, definiert wie folgt:

\begin{itemize}
	\item $Yield(a^{(0)}) = a$ f\"ur eine Konstante $a^{(0)} \in \Sigma$
	\item $Yield(f(t_1, \dots, t_n)) = Yield(t_1) \cdot \dots \cdot Yield(t_n)$
		f\"ur $f^{(n)} \in \Sigma, t_i \in T_\Sigma$
\end{itemize}

F\"ur Sprache $L$ gilt: $Yield(L) = \bigcup\limits_{ t \in \Sigma } Yield(t)$

Beispiel:
$Yield(f(f(a,b),b)) = abb$ \Tree [.f [.f a b ] b ] \\
$Yield(|| g(a, \Box, b)^{\ast . \Box} \cdot_{\Box} f*(a,b) ||) = \{ a^n b^n | n \in \mathbb{N}, n\geq 1 \}$\\

Wir wolen regul\"are Baumsprachen mit Ableitungsb\"aumen von kontextfreien Wortgrammatiken \glqq vergleichen\grqq.
Betrachte $G = (S,N,T,R)$ (kontextfreie Wortgrammatik, d.h. Regeln in $R$ haben die Form $A \to \alpha$, wobei
$A \in N, \alpha \in (N \cup T)^+$).\\
Betrachte zum Beispiel $G = (S,N,T,R)$ mit Regeln:\\
$S \to aSb, S \to ab$ mit $L(G) = \{ a^n b^n | n \in \mathbb{N}, n \geq 1 \}$.\\
Der Syntaxbaum f\"ur $aabb$ hat die Form \Tree [.S a [.s a b ] b ] , d.h. $S$ hat keinen festen Rang.\\

Betrachte daher f\"ur gegebene Grammatik $G$ Tupel $(A, m)$ f\"ur jedes $A \in N$, sodass $A \to \alpha \in R$ 
mit $|\alpha| = m$.\\
Zu einem Symbol $a \in T \cup N$ definieren wir die Menge der von $a$ ausgehenden Ableitungsb\"aume in $G$, $D(G,a)$, wie folgt:

\begin{itemize}
	\item $D(G,a) = \{ a \}$ f\"ur $a \in T$
	\item $(a,0) (\varepsilon) \in D(G,a)$, falls $a \to \varepsilon \in R$
	\item $(a,m)i (t_1, \dots, t_m) \in D(G,a)$, falls 
		$t_i \in (G,a_i), a_i \in T \cup N, a \to a_1, \dots, a_m \in R$
	\item Ableitungsb\"aume f\"ur $G$ sind $D(G) = \bigcup\limits_{a \in t \cup N} D(G, a)$
\end{itemize}

Beispiel: $G = (S,N,T,R)$ mit $S \to aSb, S \to ab$. Dann ist
\Tree [.(S,3) a [.(S,2) a b ] b ] in $D(G)$

Bemerkung: Ist $G$ eine kontextfreie Grammatik, so ist nat\"urlich $Yield(D(G)) = L(G)$.

\subsection{Satz}

1.) Ist $G = (S,N,T,R)$ eine kontextfreie Wortgrammatik, dann ist $D(G)$ regul\"are Baumsprache.\\
2.) Ist $L$ regul\"are Baumsprache, dann ist $Yield(L)$ kontextfreie Wortsprache.\\
3.) Es existieren regul\"are Baumsprachen, die nicht Ableitungsb\"aumen von kontextfreien Sprachen entsprechen.\\

Beweis:\\
1.) Erzeuge Grammatik $G' = (S,N,\Sigma,R')$ mit:\\
\begin{itemize}
	\item $\Sigma = T \cup \{\varepsilon\} \cup \{ (A,m) | A \in N, A \to \alpha \in R$ mit $|\alpha| = m \}$
	\item $A \to (A,0) (\varepsilon) \in R'$, falls $A \to \varepsilon \in R$
	\item $A \to (A,m) (A_1, \dots, A_m) \in R'$, falls $A \to A_1 \dots A_m \in R$
\end{itemize}
damit ist offensichtlich $Yield(L(G')) = L(G)$.

2.) Betrachte normalisierte Grammatik $G = (S,N,\Sigma,R)$ mit $L(G) = L$.\\
	Erzeuge Wortgrammatik $G' = (S,N,\Sigma^{(0)},R')$ mit folgenden Regeln:
\begin{itemize}
	\item $A \to A_1, \dots, A_m$ ($A \to a$) $\in R'$, falls $A \to f(A_1, \dots, A_n)$ ($A \to a$) $\in R$ f\"ur $f^{(m)} \in \Sigma$
\end{itemize}

3.) \"Ubung\\

Pfadsprache $\pi (n)$:

\begin{itemize}
	\item $a \in \Sigma ^ 0$ dann ist $\pi (a) = \{a\}$
	\item $t = f(t_1, \dots, t_n)$, dann ist $\pi (t) = \bigcup\limits_{i=1}^n \{ fiw | w \in \pi (t_i) \}$
\end{itemize}

Beispiel: $t=$ \Tree [.f [.f a b ] b ]
$\rightarrow \pi (t) = \{ f1f1a, f1f2b, f2b \}$ \\

$\pi (L) = \bigcup\limits_{t \in L} \pi (t)$

\subsection{Satz}

Ist $L \subseteq T_\Sigma$ eine regul\"are Baumsprache, dann ist $\pi(L)$ eine regul\"are Wortsprache.\\

Komplexit\"at\\

Um den Input zu formalisieren, setze: (f\"ur $t \in T_\Sigma$)

\begin{itemize}
	\item $||t|| = 1$, falls $t$ Konstante
	\item $||t|| = 1 + \sum\limits_{i = 1}^{n} ||t_i||$, falls $t = f(t_1, \dots, t_n)$\\
		$\widehat{=} |Pos(t)|$
	\item f\"ur einen Automaten $\mathcal{A}$: $||\mathcal{A}|| = |Q| + \sum\limits_{r \in \mathcal{A}} ||r||$,
		wobei $||r|| = n + 2$ f\"ur $r = f(q_1, \dots, q_n) \to q$
\end{itemize}

Betrachte  nun:\\
MEMBERSHIP
\begin{itemize}
	\item Input: $t \in T_\Sigma$
	\item Output: \glqq Yes\grqq, dgw. $t \in L(\mathcal{A})$ f\"ur gegebenen Automaten $\mathcal{A}$
\end{itemize}

$\rightarrow$ linear\\

UNIFORM MEMBERSHIP
\begin{itemize}
	\item Input: Input: $t \in T_\Sigma, \mathcal{A}$ bottom-up Baumautomat
	\item Output: \glqq Yes\grqq, dgw. $t \in L(\mathcal{A})$
\end{itemize}

$\rightarrow$ deterministischer TA: $rightarrow$ linear\\
$\rightarrow$ nichtdeterministischer TA: $rightarrow$ $O(||t|| \cdot ||\mathcal{A}||)$ (polynomiell)\\
(\glqq on the fly\grqq alle erreichbaren Zust\"ande bestimmen)\\

EMPTYNESS
\begin{itemize}
	\item Input: $\mathcal{A}$ bottom-up Baumautomat
	\item Output: \glqq Yes\grqq gdw. $L(\mathcal{A}) = \emptyset$
\end{itemize}

$\rightarrow$ \"ubersetze in Horn-Formeln\\
$\rightarrow$ linear\\

UNIVERSALITY
\begin{itemize}
	\item Input: $\mathcal{A}$
	\item Output: \glqq Yes\grqq gdw. $L(\mathcal{A}) = T_\Sigma$
\end{itemize}

$\rightarrow$ Exptime-vollst\"andig (Determinisierten, bevor $\mathcal{A}^c$ gebildet werden kann)\\

FINITENESS
\begin{itemize}
	\item Input: $\mathcal{A}$
	\item Output: \glqq Yes\grqq gdw. $L(\mathcal{A})$ endlich
\end{itemize}

$\rightarrow$ Pumping: Exptime\\
$\rightarrow$ nach Loops suchen f\"ur Automat\\
$q$ mit $C[q] \to^\ast q$ f\"ur $C \in T)\Sigma(x)$\\
und $C'[q] \to^\ast q_f \in F$ f\"ur $C' \in T)\Sigma(x)$\\
$\rightarrow$ polynomiell\\

\section{Weitere Modelle, Ausblick}

Wir betrachten 2 Modelle, die weniger ausdrucksstart als Baumautomaten sind.
Dabei gehen wir nicht mehr von einem Rangalphabet aus, d.h. interne Symbole haben
beliebig (endlich) viele Kinder.

\subsection{Definition}

Es sei $\Sigma$ eine endliche Menge (von Symbolen) und $Q$ ein Alphabet (von Bl\"attern) und
$Q \cap \Sigma = \emptyset$.
Die Menge $T_{\Sigma, Q}$ der $\Sigma$-B\"aume mit $Q$-Bl\"attern ist induktiv definiert wie folgt:
\begin{itemize}
	\item $q \in Q \Rightarrow q \in T_{\Sigma, Q}$
	\item $n \geq 1, f \in \Sigma, t_1, \dots, t_n \in T_{\Sigma, Q}
		\Rightarrow f(t_1, \dots, t_n) \in T_{\Sigma, Q}$
\end{itemize}

Die Menge $Pos(t)$ der Positionen eines Terms $t \in T_{\Sigma, Q}$ sowie Teilb\"aume etc. sind
definiert wie bisher.\\
Betrachte kontextfreie Grammatik $G = (\Sigma_0, \Sigma, Q, R)$.\\
Dann haben wir Regeln der Form $A \in \Sigma \to \alpha \in (\Sigma \cup Q)^\ast$
und erzeugen so Zeichenketten von W\"ortern.\\
Wir fassen $G$ als Baumgrammatik auf:
$A \to \alpha _1, \dots, \alpha _n \Rightarrow$ \Tree [.A $\alpha _1$ $\dots$ $\alpha _1$ ]

\subsection{Definition}

Eine lokale Baumgrammatik (LTG) ist eine kontextfreie Grammatik $G = (\Sigma _0 \Sigma, Q, R)$.
Ein Term $t \in T_{\Sigma, Q}$ wird von $G$ erzeugt, falls
\begin{itemize}
	\item $t(\varepsilon) \in \Sigma _0$
	\item Ist $w$ interene Posotion ($w \in Pos(t))$ mit $\{ i | wi \in Pos(t) \} = \{ 1, \dots, n \} \neq \emptyset$,\\
		dann ist $t(w) \to t(w_1) \cdot \dots \cdot t(w_n) \in R$
	\item Ist $w$ ein Blatt, dann ist $t(w) \in Q$
\end{itemize}

Lokale Baumsprache (LTL): $L(G) = \{ t \in T_{\Sigma, Q} | $G erzeugt t$ \}$\\

Beispiel: Sei $\Sigma _0 = \{f\}, \Sigma = \{f\}, Q = \{a,b\}, R = \{f \to fb, f \to ab\}$\\
Dann wird \Tree [.f [.f a b ] b ] von $G = (\Sigma _0, \Sigma, Q, R)$ erzeugt.\\

Offensichtlich ist $Yield(L(G))$ kontextfrei f\"ur LTG $G$.

\subsection{Satz}

Sind $G, G'$ LTGs, dann existiert eine LTG $G \cap G'$ mit
$L(G \cap G') = L(G) \cap L(G')$\\
Beweis: Betrachte $\Sigma \cap \Sigma ', Q \cap Q', R \cap R', \Sigma _0 \cap \Sigma ' _0$\\
Falls $Q \cap Q' \neq \emptyset$, ansonsten irgendeine Grammatik mit leerer Sprache, etwa $(\emptyset, \Sigma, Q, R)$\\
Beispiel: LTL sind nicht abgeschlossen unter Vereinigung\\
Betrachte Terme der Form\\
\Tree [.g [.f [.$\dots$ [.f a ] ] ] [.f [.$\dots$ [.f a ] ] ] ]\\
$\widehat{=} \{g(f^i(a),f^j(a)) | i,j \geq 1\}$\\
$\to R = \{ g \to ff, f \to f, f \to a \}$\\
$\Sigma _0 = \{g\}, Q = \{c\}$\\

Betrachte\\
\Tree [.g [.f [.$\dots$ [.f b ] ] ] [.f [.$\dots$ [.f b ] ] ] ]\\
$\widehat{=} \{g(f^i(b),f^j(b)) | i,j \geq 1\}$\\

Falls $G$ alle Terme aus $G_1$ und $G_2$ erzeugt, so auch gemischte, z.B. $g(f^i(a),f^j(b))$\\
Demzufolge sind LTBs auch nicht abgeschlossen unter Komplement.\\

Nun: Gr\"o\ss ere \glqq Bausteine\grqq f\"ur Ba\"ume

\subsection{Definition}

Eine Baumsubstitutionsgrammatik (TSG) ist ein Tupel $G = (\Sigma _0, \Sigma, Q, R)$, wobei
$R \subseteq T_{\Sigma, Q}$ (mit nicht notwendigerweise $\Sigma \cap Q = \emptyset$ eine endliche
(Menge von Fragmenten) ist.\\

Wir setzen $t \to_{TSG} t'$, falls
\begin{itemize}
	\item $p \in R$ Fragment
	\item $C \in T_{\Sigma,Q} (x)$ Kontext
\end{itemize}

existieren mit
\begin{itemize}
	\item $t = C[p(\varepsilon]$, $t' = C[p]$
\end{itemize}

Beispiel: \Tree [.f [.f a b ] b ] \\

$R = \{\Tree[.f f b ], \Tree[.f a b ]\}$\\
$\Sigma _0 = \{f\}$\\

Ist $G$ eine TSG, dann ist $L(G) = \bigcup\limits_{t_0 \in \Sigma _0} \{ t \in T_{\Sigma, Q} | t_0 \to ^\ast _TSG t, $Bl\"atter von t in Q$ \}$\\
Beispiel (Vereinigung): siehe oben, mit \glqq Pumping\grqq-Argument\\
Beispiel (Schnitt): Seien $x_i \in \{a,b\}$\\

$L_1=$ \Tree [.S $x_1$ [.S $x_1$ [.S $x_2$ [.S $x_2$ [.$\dots$ [.S $x_n$ c ] ] ] ] ] ] \\
$L_2=$ \Tree [.S $x_1$ [.S $x_2$ [.S $x_2$ [.S $x_3$ [.$\dots$ [.S $x_{n + 1}$ c ] ] ] ] ] ] \\
$L_1 \cap L_2=$ \Tree [.S $x_1$ [.S $x_1$ [.S $x_1$ [.S $x_1$ [.$\dots$ [.S $1$ c ] ] ] ] ] ] \\

\end{document} 
