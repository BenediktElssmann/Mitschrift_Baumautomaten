
\documentclass[titlepage]{article}

%Paket-Liste
\usepackage{geometry}
\usepackage[german]{babel}
\usepackage{mathtools}
\usepackage{stmaryrd}
\usepackage{qtree}
\usepackage{amssymb}
\usepackage{amsmath}
\usepackage{tikz}
\usetikzlibrary{automata,positioning}

\setcounter{section}{-1}
\setlength{\parindent}{0pt}


\begin{document}

\title{\huge{Vorlesung Baumautomaten (Mitschrift)}}
\author{Benedikt El\ss mann (3720358)\\be57xocu@studserv.uni-leipzig.de\\\\Universit\"at Leipzig}

\maketitle

\thispagestyle{empty}

\tableofcontents

\newpage

\section{Einleitung}

Automaten lesen W\"orter $ w = a_1 \dots a_n $ und geben ''accept'' aus oder nicht.\\
Daf\"ur gibt es Erweiterungen, wie etwa:

\begin{itemize}
	\item gewichtete Automaten, das hei\ss t der Output ist ein Semiringelement
	\item Automaten mit Ged\"achtnis (Stack)
	\item Automaten \"uber anderen  Strukturen

	\begin{itemize}
		\item $\omega$-W\"orter $ w = a_1 \dots a_n $ 
		\item Graphen
		\item B\"aume\\
		\item Kombinationen dieser
	\end{itemize}

\end{itemize}

Typische Fragestellungen:

\begin{itemize}
	\item Ausdrucksst\"arke
	\item Darstellung als rationale Ausdr\"ucke (Kleene)
	\item Darstellung als Grammatik
	\item Darstellung als Logik
\end{itemize}

\section{B\"aume und Baumautomaten}

Wir betrachten \"uber $ A = \{a,b\} $ den Automaten $ \mathcal{A} $:

\begin{tikzpicture}[shorten >= 1pt, node distance=2cm, on grid, auto]
	\node[state, initial] (q0) {$ q_0 $};
	\node[state] (q1) [right=of q0 ] {$ q_1 $};
	\node[state, accepting] (q2) [right=of q1 ] {$ q_2 $};

	\path[->]	(q0)	edge 					node	{a}	(q1);
	\path[->]	(q0)	edge 	[loop above] 	node 	{b}	();
	\path[->]	(q1)	edge					node 	{b}	(q2);
	\path[->]	(q2)	edge	[loop above]	node	{a}	();
\end{tikzpicture}

mit $ L(\mathcal{A}) = b^\ast ab a^\ast $ .

Betrachtung des Wortes $ w = baba \in L(\mathcal{A}) $:\\

Der eindeutige erfolgreiche Lauf f\"ur $ w $ l\"asst sich darstellen als:\\

$q_0 baba \rightarrow b q_0 aba \rightarrow ba q_1 ba \rightarrow bab q_2 a \rightarrow baba q_2 \in F$ (Finalzustand)\\

Baumautomaten funktionieren analog. Unser erstes Beispiel wird

\Tree [.f [.f a b ] [.b ] ]

Akzeptiert mit dem Lauf:

\Tree [.f [.f a b ] [.b ] ] $\rightarrow$
\Tree [.f [.f $q_a$ $q_b$ ] [.$q_a$ ] ] $\rightarrow$
\Tree [.f [.$q_a$ $q_a$ $q_b$ ] [.$q_a$ ] ] $\rightarrow$
\Tree [.$q_f$ [.$q_a$ $q_a$ $q_b$ ] [.$q_a$ ] ]\\

mit $q_f \in F$ 

\subsection{Definition Rangalphabet}

Ein paar $(\Sigma, rk)$, wobei $\Sigma$ eine endliche Menge von Symbolen und $rk : \Sigma \rightarrow \mathbb{N}$ eine Abbildung ist, hei\ss t Rangalphabet.\\
F\"ur $f\in \Sigma$ hei\ss t $rk(f)$ der Rang (oder die Stelligkeit) von $f$.\\ \\
Intuitiv: $rk(f)$ ist die Anzhal der Kinder von $f$ in einem Baum.\\
Insbesondere ist die Anzahl der Kinder f\"ur jedes Symbol fest.\\ \\
Gilt $rk(f) = n$, schreiben wir auch $f^{(n)}$ statt $f$. wir schreiben:

\begin{itemize}
	\item 0-stellige Symbole (Konstanten) $a, b, \dots$
	\item un\"ar, bin\"ar, \dots $f, g, \dots$\\
\end{itemize}

Wir setzen $\Sigma^{(n)} = \{f \in \Sigma | rk(f) = n\}$\\

In 
\Tree [.f [.f a b ] [.b ] ] 
ist also $rk(f) = 2, rk(b) = 0$\\

\subsection{Definition Term, Tree}

Sei $(\Sigma, rk)$ ein Rangalphabet. Die Menge $T_{\Sigma}$ der B\"aume \"ueber $\Sigma$ ist induktiv definiert durch:

\begin{itemize}
	\item $\Sigma^0 \subseteq T_{\Sigma}$
	\item $f^{(n)} \in \Sigma$ . $t_1, \dots, t_n \in T_{\Sigma}$, dann ist $f(t_1, \dots, t_n) \in T_{\Sigma}$
\end{itemize}

Intuitiv sind $t_1, \dots, t_n$ die Kinder von $f$.\\
Z.B. ist \Tree [.f [.f a b ] [.b ] ] der Term $f(f(a,b),b)$.

\subsection{Definition H\"ohe}

Sei $(\Sigma, rk)$ ein Rangalphabet. Die H\"ohe $ht$ ist gegeben durch:

\begin{itemize}
	\item f\"ur $a^{(0)} \in \Sigma: ht(a) = 1$.
	\item f\"ur $f(t_1, \dots , t_n) \in T_{\Sigma} : ht(f) = 1 + max\{ ht(t_i) | i \in \{i, \dots, n\}\}$
\end{itemize}

Ziel: Zugriff auf einen Knoten innterhalb eines Baumes und deren Label.\\
Daf\"ur ordnen wir den Knoten Positionen zu. Das geht induktiv wie folgt:

\subsection{Definition Position}

Sei $(\Sigma, rk)$ ein Rangalphabet. Die Positionenmenge ist definiert durch:

\begin{itemize}
	\item f\"ur $a^{(0)} \in T_{\Sigma}$ ist $Pos(a) = \{\varepsilon\}$
	\item f\"ur $f(t_1, \dots, t_n) \in T_{\Sigma}$ ist $Pos(f(t_1, \dots, t_n)) = \{\varepsilon\}1 \cdot Pos(t_1) \cup \dots \cup n \cdot Pos(t_n)$
\end{itemize}

Beispiel:\\
Betrachtung von $f(f(a,b),b)$ bzw. \Tree [.f [.f a b ] [.b ] ] der Term $f(f(a,b),b)$\\

$Pos(f) = \{\varepsilon, 1, 2, 1.1, 1.2\}$

\subsection{Definition der Label an den Positionen}

F\"ur einen Term der Form $t = f(t_1, \dots, t_n)$ ist das Symbol $t(p)$ in $t$ an p-ter Position induktiv definert durch:

\begin{itemize}
	\item $t(\varepsilon) = f$
	\item $t(ip) = t_i(p), i\in \{1, \dots , n\}$
\end{itemize}

Beispiel: Betrachtung von $f(f(a,b),b)$\\
Dann ist\\
$t(\varepsilon) = f$\\
$t(1) = t(1 \cdot \varepsilon) = t_1(\varepsilon) = f$\\
$t(2) = t(2 \cdot \varepsilon) = t_2(\varepsilon) = b$\\
$t(1.1) = t_1(1) = a$\\
$t(1.2) = t_2(1) = b$

\subsection{Definition Sub-Baum}

F\"ur $T_{\Sigma}$ ist ein Sub-Baum $t_{|p}$ an p-ter Position wie folgt definiert:

\begin{itemize}
	\item $Pos(t_{|p}) = \{ i | pi \ \in Pos(t)\}$
	\item $\forall q \in Pos(t_{|p}$ ist $t_{|p}(q) = t(pq)$
\end{itemize}

Wir schreiben $t[u]_p$ f\"ur den Baum, der entsteht, wenn man in $t$ den sub-Baum $t_{|p}$ durch $n$ ersetzt.\\

Beispiel: $f(f(a,b),b)$ bzw. \Tree [.f [.f a b ] [.b ] ]

$t_{|1} = f(a,b)$ \Tree [.f a b ]\\

$t_{|2} = t(1.2) = b$\\

$u = g(b,a)$ \Tree [.g b a ], dann ist\\

$t[u]_{|1} = f(g(b,a),b)$ \Tree [.f [.g b a ] [.b ] ]

\subsection{Definition Baumautomat}

Ein Buamautomat $ \mathcal{A}$ ist ein 4-Tupel $(Q,\Sigma,F,\Delta)$, wobei:\\
$Q \dots$ endliche Menge an Zus\"anden\\
$\Sigma \dots$ Rangalphabet, wobei $\Sigma \cup Q \neq \emptyset$\\
$F \dots \subseteq Q$ Finalzust\"ande\\
$\Delta \dots$ Menge von Regeln\\

$r: f(q_1 \dots q_n) \rightarrow q$\\
f\"ur $q, q_1, \dots , q_n \in Q$, f\"ur $a^{(0)} \in T_{\Sigma} : a \rightarrow q$\\

Beispiel:\\
$\mathcal{A} = \{\{q_a, q_b, q_f\},\{a^{(0)},b^{(0)},f^{(2)}\},\{q_f\}, \Delta\}$ \\
mit $\Delta = \{ a \rightarrow q_a, b \rightarrow q_b, f(q_a,q_b) \rightarrow q_a, f(q_a, q_b), f(q_a,q_b) \rightarrow q_f\}$

\subsection{Definition Lauf/Run}

Sei $\mathcal{A} = (Q, \Sigma, F, \Delta)$ ein Baumautomat und $t\in T_{\Sigma}$. 
Ein Lauf r f\"ur $t$ von $\mathcal{A}$ ist ein Term mit

\begin{itemize}
	\item $Pos(r) = Pos(t)$
	\item Ist $t(p) = a$ ein Blatt, dann ist $r(p) = q_a$, nur wenn $(a\rightarrow q_a) \in \Delta$ 
	\item Ist $t(p)=f^{(m)}$, dann ist $r(p) = q$, wenn $(f(q_1, \dots, q_n) \rightarrow q) \in \Delta$ und $r(p_i) = q_i$, $i \in \{1, \dots, n\}$
\end{itemize}

Ein Lauf ist erfolgreich, wenn $r(\varepsilon) \in F$. Der Automat $\mathcal{A}$ akzeptiert $t$, falls es einen erfolgreichen Lauf
f\"ur $t$ von $\mathcal{A}$ gibt.\\
Wir bezeichnen mit $L(\mathcal{A}) = \{t \in T_{\Sigma} | \mathcal{A}$ akzeptiert $t\}$ die von $\mathcal{A}$ erkannte Baumsprache. Eine Sprache $L \subseteq T_{\Sigma}$ hei\ss t erkennbar, falls ein Baumautomat $\mathcal{A}$ existiert mit $L=L(\mathcal{A})$.\\

Um einzelne Schritte von Baumautomaten zu formalisieren, betrachten wir die \textit{move relation} $\rightarrow_{\mathcal{A}}$, 
definiert wie folgt:\\
Gegeben sei $\mathcal{A} = (Q, \Sigma, F, \Delta)$, dann ist $t \rightarrow _{\mathcal{A}} t'$ mit $t, t' \in T_{\Sigma \cup Q}$, falls

\begin{itemize}
	\item $t(p) = f^{(n)}$
	\item $t(pi) = q_i$ f\"ur $i \in \{1, \dots, n\}$ und $p_i$ sind Bl\"atter
	\item $(f(q_1, \dots, q_n) \rightarrow q) \in \Delta$
	\item und $t' = t[q]_p$
\end{itemize}

Mit $\rightarrow^\ast_{\mathcal{A}}$ bezeichnen wir die transitive H\"ulle von $\rightarrow_{\mathcal{A}}$.

\subsection{Lemma}

Sei $\mathcal{A} = (Q, \Sigma, F, \Delta)$ ein Baumautomat. Dann ist 
$L(\mathcal{A}) = \{t \in T_{\Sigma} | t \rightarrow^\ast_{\mathcal{A}} q$ mit $q \in F\} (=Z)$\\

Beweis: \glqq$L(\mathcal{A}) \subseteq Z$\grqq:\\

Wir zeigen: Es existiert ein Run $r$ f\"ur $t$ von $\mathcal{A}$ mit $r(\varepsilon) = q$, dann ist $t \rightarrow ^\ast_{\mathcal{A}} q$\\ \\
Inuktionsannahme:\\
$t=a^{(0)} \in T_{\Sigma}$. Dann gilt $a \in L({\mathcal{A}})$, falls ein Lauf $r$ existiert mit $r(a) = q_a$ und $(a \rightarrow q_a) \in \Delta$. Dann folgt $a \rightarrow^\ast_{{\mathcal{A}}} q_a$.\\
Sei nun $t=f(t_1, \dots, t_n)$\\ \\

Induktionsvoraussetzung:\\
Falls f\"ur $t_1, \dots , t_n$ L\"aufe $r_i$ existieren mit $r_i(\varepsilon) = q_i$, dann gilt auch $t_i \rightarrow ^\ast_{\mathcal{A}} q_i$ mit $i\in \{1, \dots , n\}$\\ \\

Induktionsschritt:\\
zu zeigen: Es existiert ein Lauf $r$ f\"ur $t$ mit $r(\varepsilon) = q$, dann $t \rightarrow ^\ast_\mathcal{A} q$.\\
Sei also $r$ ein Lauf mit $r(\varepsilon) = q$. Dann ist $r(i) = q_i$, $i \in \{1, \dots, n\}$, mit
$(f(q_1, \dots, q_n) \rightarrow q) \in \Delta$. Laut Induktionsvoraussetzung gilt nun,
$t_i \rightarrow ^\ast_\mathcal{A} q_i$, $i \in \{1, \dots, n\}$.\\
Damit $t = f(t_1, \dots, t_n) 
\rightarrow ^\ast_\mathcal{A} f(q_1, t_2, \dots, t_n)
\rightarrow ^\ast_\mathcal{A} \dots
\rightarrow ^\ast_\mathcal{A} f(q_1, \dots, q_n)$\\
Des weiteren haben wir die regel $f(q_1, \dots, q_n) \rightarrow q$, das hei\ss t $f(q_1, \dots, q_n) \rightarrow ^\ast_\mathcal{A} q$.\\ \\

Insgesamt also $t \rightarrow ^\ast_\mathcal{A} q$ \\

Beweis: \glqq$L(Z \subseteq \mathcal{A})$\grqq: analog\\ \\

Einige Beispiele f\"ur Baumautomaten:\\

1. Sei $B = (\{q_0, q_1\}, \{0^{(0)},1^{(0)},\lnot^{(1)},\land^{(2)},\lor^{(2)}\,\{q_1\},\Delta\}$ mit\\
$\Delta = \{ 0 \rightarrow q_0, 1 \rightarrow q_1,\\
\lnot (q_0) \rightarrow q_1, \lnot (q_1) \rightarrow q_0,\\
\land (q_0, q_0) \rightarrow q_0, \land (q_0, q_1) \rightarrow q_0, \land (q_1, q_0) \rightarrow q_0, \land (q_1, q_1) \rightarrow q_1\\
\lor (q_0, q_0) \rightarrow q_0, \lor (q_0, q_1) \rightarrow q_1, \lor (q_1, q_0) \rightarrow q_1, \lor (q_1, q_1) \rightarrow q_1\}$\\

Beispiellauf:\\

\Tree [.$\lnot$ [.$\land$ [.$\lor$ 0 [.$\lnot$ 1 ] ] [.$\lor$ 1 [.$\land$ 0 0 ] ] ] ] $\rightarrow$
\Tree [.$\lnot$ [.$\land$ [.$\lor$ $q_0$ [.$\lnot$ $q_1$ ] ] [.$\lor$ $q_1$ [.$\land$ $q_0$ $q_0$ ] ] ] ] $\rightarrow$
\Tree [.$\lnot$ [.$\land$ [.$\lor$ $q_0$ [.$q_0$ ] ] [.$\lor$ $q_1$ [.$\land$ $q_0$ $q_0$ ] ] ] ] $\rightarrow$ \\ \\
\Tree [.$\lnot$ [.$\land$ [.$\lor$ $q_0$ $q_0$ ] [.$\lor$ $q_1$ $q_0$ ] ] ] $\rightarrow$
\Tree [.$\lnot$ [.$\land$ $q_0$ $q_1$ ] ] $\rightarrow$
\Tree [.$\lnot$ $q_0$ ] $\rightarrow$
\Tree [.$q_1$ ] \\ \\

2. $(a^nb^n light)$\\

Betrachten $\mathcal{A} = (\{q_a, q_b, q_f\}, \{a^{(0)}, b^{(0)}, f^{(3)}, g^{(2)}\}, \{q_f\}, \Delta)$\\
mit $\Delta = \\
\{a \rightarrow q_a, b \rightarrow q_b, g (q_a, q_b) \rightarrow q_f, f (q_a, q_f, q_b) \rightarrow q_f\}$\\

Beispiellauf:\\

\Tree [.$f$ $a$ [.$f$ $a$ [.$g$ $a$ $b$ ] $b$ ] $b$ ] $\rightarrow$
\Tree [.$f$ $q_a$ [.$f$ $q_a$ [.$g$ $q_a$ $q_b$ ] $q_b$ ] $q_b$ ] $\rightarrow$
\Tree [.$f$ $q_a$ [.$f$ $q_a$ $q_f$ $q_b$ ] $q_b$ ] $\rightarrow$
\Tree [.$f$ $q_a$ $q_f$ $q_b$ ] $\rightarrow$
\Tree [.$q_f$ ] \\

$\mathcal{A}$ akzeptiert also alle B\"aume der Form:\\
\Tree [.$f$ $a$ [.$f$ $a$ [.... [.$g$ $a$ $b$ ] ] $b$ ] $b$ ]\\

3. Simulation eines Wortautomaten: (siehe \"Ubung)\\

Betrachtet man $\Sigma =\{a^{(0)}, f^{(2)}, g^{(1)}\}$. Dann ist $L = \{f(g^i(a), g^i(a)) | i \geq 0\}$ nicht erkennbar.

\subsection{Definition Determinismus}

Ein Automat $\mathcal{A} (Q, \Sigma,  F, \Delta)$ hei\ss t deterministisch, falls: aus
$ f(q_1, \dots, q_n) \rightarrow q $ und \\
$ f(q_1, \dots, q_n) \rightarrow q' $ folgt $ q = q'$

\subsection{Satz}

Sei $\mathcal{A} = (Q, \Sigma, F, \Delta)$ ein Baumautomat, dann existiert ein deterministischer Baumautomat $\mathcal{A}_d$, 
so dass $L(\mathcal{A}) = L(\mathcal{A}_d)$.\\ \\

Beweis: Setze $\mathcal{A}_d = (Q_d, \Sigma, F_d, \Delta_d)$ \\
mit $Q_d = 2^Q$ $(\ast)$\\
und $f(s_1, \dots, s_n) \rightarrow s \in \Delta_d$\\
$\Leftrightarrow s = \{ q \in Q | \exists q_1 \in s_1 \dots q_n \in s_n: (f(q_1, \dots, q_n) \rightarrow q) \in \Delta\}$ \\
und $F_d = \{ s \in Q_d | s \cap F \neq \emptyset\}$.\\ \\
Wir zeigen:\\
1. $\mathcal{A}$ ist deterministisch \\
2. $L(\mathcal{A}) \subset L(\mathcal{A}_d)$ \\
3. $L(\mathcal{A}_d) \subset L(\mathcal{A})$ \\ \\

1. ist klar, denn $(\ast)$ ist mit einer \"Aquivalenz definiert.\\
2. \glqq$L(\mathcal{A}) \subseteq L(\mathcal{A}_d)$\grqq:\\

Wir zeigen hierzu: Ist $Z = \{ q | t \rightarrow ^\ast_\mathcal{A} q\}$, dann $ t \rightarrow ^\ast_{\mathcal{A}_d} z$.\\ \\

Induktionsannahme:\\
Angenommen $a \rightarrow _\mathcal{A} q_a$, dann ist $q_a \in \{ q \in Q | q \rightarrow ^\ast_\mathcal{A} q\}$, das hei\ss t\\
$a \rightarrow ^\ast_\mathcal{A} q_a \Leftrightarrow q_a \in \{ q \in Q | a \rightarrow ^\ast_\mathcal{A} q \}$\\
$\Leftrightarrow q_a \in \{ q \in Q | (a \rightarrow q) \in \Delta \}$\\
also $a \rightarrow ^\ast_\mathcal{A} q_a \Leftrightarrow q_a \in \{ q \in Q | (a \rightarrow q) \in \Delta \}$, das hei\ss t\\
$z := \{ q_a \in Q | a \rightarrow ^\ast_\mathcal{A} q_a \} = \{ q \in Q | (a \rightarrow q) \in \Delta \} =: s$\\

Nun ist $( a \rightarrow s) \in \Delta_d$ per Definition, also auch $( a \rightarrow z) \in \Delta_d$, damit: $a \rightarrow ^\ast_{\mathcal{A}_d} z$.\\

Betrachten wir nun $ t = \sigma(t_1, \dots , t_n) $ \\

Induktionsvoraussetzung:\\
$t_i \rightarrow ^\ast_{\mathcal{A}_d} z_i$ mit $Z_i = \{q \in Q | t_i \rightarrow ^\ast_\mathcal{A} q\}$\\
Das hei\ss t, es existieren L\"aufe $r_i$ f\"ur $t_i$ von $\mathcal{A}_d$ mit $r_i(\varepsilon) = z_i$\\

Induktionsschritt:\\
zu zeigen: $ t \rightarrow ^\ast_\mathcal{A} z$ mit $Z = \{ q \in Q | t \rightarrow ^\ast_\mathcal{A} q \} $ \\
Das hei\ss t, es existiert ein Lauf $r$ f\"ur $t$ von $\mathcal{A}_d$ mit $r(\varepsilon) = z$\\
Das hei\ss t, $\exists r :$\\

\begin{itemize}
	\item $r (\varepsilon) = z$
	\item $r (i) = z_i$
	\item $\sigma (z_1, \dots, z_n) \rightarrow z \in \Delta_d$
\end{itemize}

Setze nun $r_{ |i } = r_i$, damit ist insbesondere $r(i) = r_i(\varepsilon) = z := \{q | t_i \rightarrow ^\ast_{\mathcal{A}_d} q \}$\\

Es bleibt also zu zeigen: $ \exists$ Regel $\sigma (z_i, \dots, z_m) \rightarrow z \in \Delta_d$.\\
Es ist nun $ z \in Z \Leftrightarrow t \rightarrow ^\ast_\mathcal{A} z$ \\
$\Leftrightarrow \exists q_i \in Q: t_i \rightarrow ^\ast_\mathcal{A} q_i, \sigma(q_1, \dots, q_m) \rightarrow z \in \Delta$\\
$\Leftrightarrow \exists z_i \in Z_i$ und $\sigma/z_1, \dots , z_m) \rightarrow z \in \Delta$ \\
Also $Z= \{z\in Q | \exists z_i \in Z_i : (\sigma/z_1, \dots , z_m) \rightarrow z ) \in \Delta)$ also per Definition
$\sigma/z_1, \dots , z_m) \rightarrow z \in \Delta_d$\\ \\

2. \glqq$L(\mathcal{A}_d) \subseteq L(\mathcal{A})$\grqq:\\

Sei $t \in T_\Sigma$ mit $t \notin L(\mathcal{A})$, dann ist $Z \cap F = \{ q \in Q | t \rightarrow^\ast_\mathcal{A} q \} \cap F = \emptyset$\\
Laut 2. ist $t \rightarrow ^\ast_{\mathcal{A}_d} z$ (und $\mathcal{A}$ ist deterministisch)
Wegen $Z \cap F = \emptyset$ ist $Z \notin F_d$, also $t \notin L(\mathcal{A}_d)$ \\ \\

Wir vereinbaren die Abk\"urzungen: NBA/NTA f\"ur nichtdeterministischer Baumautomat und DBA/DTA f\"ur deterministischer Baumautomat.\\
Wie im Wortfall ist die Konstruktion exponentiell, das hei\ss t wir ben\"otigen expontntiell viele Zust\"ande ($Q_d = 2^{|Q|}$).
Und wie im Wortfall l\"asst sich das im Allgemeinen nicht vermeiden.\\ \\
Beispiel: Betrachtet man $\Sigma = \{ f^{(1)}, g^{(1)}, a^{(0)}\}$ und
sei $L_n = \{ f \in T_\Sigma | t( \underbrace{1 \dots 1}_\text{n-mal} ) = f \}$ \\

Ein NTA ben\"otigt $n+2$ Zust\"ande:\\
$\mathcal{A} = (Q, \Sigma, F, \Delta)$ mit $Q = \{q, q_1, \dots , q_{n+1} \}$, $F = \{q_{n+1}\}$ \\
mit \"Uberg\"angen
$\Delta = \{a \rightarrow q, f(q) \rightarrow q, g(q) \rightarrow q, f(q) \rightarrow q_1, \\
f(q_i) \rightarrow q_{i+1}, g(q_i) \rightarrow q_{i+1} \}$ f\"ur $i \in \{1, \dots ,n\}$\\

Man kann zeigen: Ein DTA $\mathcal{A}'$ mit $L(\mathcal{A}') = L_n$ hat mindestens $2^{n+1}$ Zust\"ande.

\subsection{Definition vollst\"anding und reduziert}

Ein Automat $(\mathcal{A} = Q, \Sigma, F, \Delta)$ hei\ss t:

\begin{itemize}
	\item vollst\"andig, falls f\"ur jedes $f^{(n)} \in \Sigma$ und alle $q_1, \dots , q_n \in Q$ eine Regel 
		$f(q_1, \dots, q_n) \rightarrow q \in \Delta$ existiert.
	\item reduziert, falls f\"ur jeden Zustand $q \in Q$ ein Term $t \in T_\Sigma$ exisitert mit $f \rightarrow ^\ast_\mathcal{A} q$
\end{itemize}

\subsection{Satz}

Sei $\mathcal{A}$ ein Baumautomat. Dann existiert ein vollst\"andiger, reduzierter Baumautomat 
$\mathcal{A}'$ mit $L(\mathcal{A}) = L(\mathcal{A}')$.\\

F\"ur Wortautomaten gibt es das Pumping-Lemma, das die Ged\"achtnislosigkeit der Automaten formalisiert.
Formal besagt es: Ist $L$ eine regul\"are Wortsprache, dann existiert ein $n \in \mathbb{N}$, so dass sich $w \in L$ mit $|w| > n$ 
zerlegen l\"asst in $w = xyz$, $y \neq \varepsilon$ und $\forall i \geq 0$ ist $xy^iz \in L$. \\ \\

Baumautomaten haben auch kein Ged\"achtnis, also erwarten wir ein analoges Resultat.
Dazu m\"ussen wir formalisierten, was \glqq aufgepumpt \grqq werden soll.

\subsection{Definition Kontext}

Es sei $\Sigma$ ein Rangalphabet und $x^{(0)} \notin \Sigma$. Es sei $C \in T_{\Sigma \cup \{x\}}$.
Falls es genau eine Position $p \in Pos(C)$ gibt mit $C(p) = x$, dann hei\ss t $C$ ein Kontext.\\

Beispiel: \Tree [.f [.f x b ] [.b ] ] ist ein Kontext.\\

Wir schreiben $T_\Sigma(x)$ f\"ur die Menge aller solcher Kontexte.\\

Ist $C \in T_\Sigma(x)$ mit $C(p) = x$, dann schreiben wir $C[u]$ statt $C[u]_p$ f\"ur den Baum, der entsteht, 
wenn wir $x$ durch $u$ ersetzen.\\
Wir schreiben $C^0 = x$, $C^1 = C$, $C^n = C^{n-1} [C]$\\

Beispiel: Betrachtet $t = $ \Tree [.f [.f a b ] [.b ] ]\\

Setze $u = f(a,b)$ und $C = f(x,b)$.\\
Dann ist $t = C[u]$ und $C^2[u] = $
\Tree [.f [.f [.f a b ] b ] [.b ] ]

\subsection{Pumping-Lemma}

Sei $L \subseteq t_\Sigma$ erkennbar, dann existiert ein $k \in \mathbb{N}$, so dass:\\
F\"ur alle $T \in L$ mit $ht(t) > k$ gibt es einen Kontext $C \in T_\Sigma (x)$, einen nicht-trivialen Kontext $C' \in T_\Sigma (x)$
und einen Term $u \in T_\Sigma$ mit $t = C[C'[u]]$ und $C[(C')^n[u]] \in L$ f\"ur alle $n \ge 0$.\\ \\

Beweis: Sei $L$ erkennbar, das hei\ss t $\exists$ Baumautomat $\mathcal{A} = (Q, \Sigma, F, \Delta)$ mit $L = L(\mathcal{A})$.
Setze $|Q| = k$ und betrachte $t \in L$ mit $ht(t) > k$. Betrachte nun einen Lauf $r$ und einen Pfad in $t$, der l\"anger als $k$ ist.
Nun gibt es $p_1,p_2 \in Pos(r)$ mit $r(p_1) = r(p_2) = q \in Q$.
Sei nun $u = t_{|p_2}$ der Sub-Baum von $t$ bei $p_2$ und $u' = t_{|p_1}$.
Dann existiert $C'$ mit $C'[u] = u'$ und es existiert $C$ mit $t=C[C'[u]]$.
Es ist wegen $t \in L$\\
$C[C'[u]] \rightarrow ^\ast_\mathcal{A} C[C'[q]] \rightarrow ^\ast_\mathcal{A} C[q] \rightarrow ^\ast_\mathcal{A} q_f \in F$, also auch\\
$C[(C')^n[u]] \rightarrow ^\ast_\mathcal{A} C[(C')^n[q]] \rightarrow ^\ast_\mathcal{A} CC[(C')^{(n-1)}[q]] 
\rightarrow ^\ast_\mathcal{A} \dots \rightarrow ^\ast_\mathcal{A} C[q] \rightarrow ^\ast_\mathcal{A} q_f \in F$. q.e.d.\\ \\

Beispiel: Betrachte den Baumautomaten 
$\mathcal{A} = (\{q_a, q_b, q_g, q_f\}, \{a^{(0)}, b^{(0)}, g^{(1)}, f^{(2)}\}, \{q_f\}, \Delta)$ mit\\
$\Delta = \{ a \rightarrow q_a, b \rightarrow q_b, f(q_a, q_b) \rightarrow q_g, g(q_g) \rightarrow q_f\}$ \\

\Tree [.g [.f [.f [.f a b ] b ] b ] ] \\
$u = f(a, b)$, $u' = C'[u] = f(f(a,b),b)$ \\
$C = g(f(x, b))$, $C' = f(x, b)$ \\

$C[(C')^n[u]] = $ \Tree [.g [.f [.f [.... [.f a b ] ] b ] b ] ] \\

Die Sprache $L = \{ f(g^i(a),g^i(a)) | i \geq 0\}$ kann nicht erkennbar sein, denn f\"ur gro\ss e $i$ w\"urde man ein $k$ finden,
so dass ein gegebener Baumautomat auch $f(g^{i+lk}(a), g^i(a))$ für alle $l \geq 0$ akzeptiert.

\subsection{Korollar}

F\"ur $\mathcal{A} = (Q, \Sigma, F, \Delta)$ ist $L(\mathcal{A}) \neq \emptyset \Leftrightarrow \exists t \in L$ mit 
$ht(t) \leq |Q|$:

\begin{itemize}
	\item $L(\mathcal{A}) |$ nicht endlich $\Leftrightarrow \exists t \in L$ mit $|Q| < ht(t) \leq 2 |Q|$
\end{itemize}

\subsection{Abschlusseigenschaften}

Erkennbare Sprachen sind abgeschlossen unter Vereinigung, Schnitt und Komplement. Das hei\ss t, sind $L_1$ und $L_2$ erkennbar, dann auch
$L_1 \cup L_2$, $L_1 \cap L_2$ und $L_1^c$ (in $T_\Sigma$).\\

Beweis:\\
Seien $\mathcal{A}_1$ und $\mathcal{A}_2$ vollst\"andige DTA. Betrachte f\"ur die Vereinigung\\
$\mathcal{A}_\cup = (Q_1 \times Q_2, \Sigma, F_1 \times Q_2 \cup Q_1 \times F_2, \Delta_1 \times \Delta_2)$ mit \\
$\Delta_1 \times \Delta_2 = \{ f((q_1, q'_1), \dots, (q_n, q'_n)) \rightarrow (q, q') |
f(q_1, \dots, q_n) \rightarrow q \in \Delta_1, f(q'_1, \dots, q'_n) \rightarrow q' \in \Delta_2 \}$

Dann akzeptiert $\mathcal{A}_\cup$ die Sprache $L(\mathcal{A}_1) \cup L(\mathcal{A}_2)$.\\ \\

F\"ur $L(\mathcal{A}_1) \cap L(\mathcal{A}_2)$ betrachte den Automaten

$\mathcal{A}_\cap = (Q_1 \times Q_2, \Sigma, F_1 \times F_2, \Delta_1 \times \Delta_2)$\\

F\"ur $T_\Sigma \ L(\mathcal{A}_1) = L(A_1)^c$ betrachte

$\mathcal{A}_C = (Q_1, \Sigma, Q_1 \ F_1, \Delta_1)$.

Der Automat $A_C$ akzeptiert $L(A_1)^c$.\\

Beispiel:\\

Betrachte $\Sigma = \{ a^{(0)}, g^{(1)}, f^{(2)}\}$ und 
$L = \{ f(g^i(a),g^j(a)) | i \leq j \}$

Dann ist $L$ nicht erkennbar, denn: W\"are $L$ erkennbar, dann auch $L'$ mit $L' = \{ f(g^i(a), g^j(a) | i \geq j \}$, 
also auch $L \cap L' \lightning$. \\

Bemerkung: Wenn $\mathcal{A}$ deterministisch und vollst\"andig ist, dann k\"onnen wir eine \"Ubergangsfunktion 
$\delta: T_\Sigma \rightarrow Q$ definieren mit $\delta (t) = q$, fals $t \rightarrow_\mathcal{A}^\ast q$.\\

Wiederholung - \"Aquivalenzrelation:\\

Eine \"Aquivalenzrelation $\sim $ auf einer Menge $M$ ist eine Relation mit

\begin{itemize}
	\item $\forall m \in M: m\sim m$
	\item $\forall m, n \in M: m\sim n \Rightarrow n\sim m$
	\item $\forall l, m, n \in M: l\sim m, m\sim n \Rightarrow l\sim n$
\end{itemize}

Insbesondere: Ist $\sim $ eine \"Aquivalenzrelation auf $M$, so induziert $\sim $ eine Partition auf und umgekehrt, das hei\ss t Mengen 
$(M_i)_{i \in I}$ mit $M_i \cup M_j = \emptyset$ f\"ur $i \neq j$ und $M = \underset{i \in I}{\cup} M_i$

\subsection{Definition Kongruenz}

Eine \"Aquivalenzrelation $\equiv$ auf $T_\Sigma$ hei\ss t Kongruenz, falls f\"ur alle $f^{(n)} \in \Sigma$: \\

$v_1 \equiv u_1, \dots, v_n \equiv u_n \Rightarrow f(v_1, \dots, v_n) \equiv f(u_1, \dots, u_n)$.

Beispiel:\\
Die Relation $t \equiv t'$, falls $t$ und $t'$ die gleiche Anzahl Bl\"atter modulo 2 haben.

\begin{itemize}
	\item Au\ss erdem: $t \equiv t' \Leftrightarrow ht(t) = ht(t')$
	\item Nicht: gleiche H\"ohe modulo 2
\end{itemize}

\subsection{Definition}

Eine Kongruenz $\equiv$ hat endlichen Index, falls $\equiv$ endlich viele \"Aquivalenzklassen indiziert.

\subsection{Lemma}

Sei $\Sigma$ ein Rangalphabet. Dann ist $\equiv$ genau dann eine Konguenz auf $T_\Sigma$, wenn $\equiv$ eine \"Aquivalenzrelation ist mit
$u \equiv v \Rightarrow C[u] \equiv C[v]$ f\"ur alle Kontexte.\\

Beweis:\\
\glqq $\Rightarrow$ \grqq Induktion:\\
Induktionsannahme: $C = x$, dann ist $u \equiv v \Rightarrow C[u] \equiv C[v]$ klar.\\
Sei nun $C = f(C_1, \dots C_n)$. Sei $x = C[ip] = C_i[p]$.\\
Dann ist $C[u]_{ip} = f(C_1, \dots, C_{i-1}, C_i[p], C_{i+1}, \dots, C_n) = C[u]_{ip} = f(C_1, \dots, C_{i-1}, C_i[v], C_{i+1}, \dots, C_n)$
\\

\glqq $\Leftarrow$ \grqq: Angenommen $u \equiv v$ und $C[u] \equiv C[v]$ f\"ur alle Kontexte.\\
Sei $f^{(n)} \in \Sigma$. Dann ist:\\
$f(u_1, \dots u_n)\\
= C^1[u_1] \equiv C^1[v_1] = f(v_1, u_2, \dots, u_n)\\
= \dots\\
= C^1[u_n] \equiv C^1[v_n] = f(v_1, \dots, v_n)$

Betrachte nun eine Sprache $L \subseteq T_\Sigma$ von B\"aumen. Wir definieren $\equiv_L$ als:\\
$u \equiv_L v \Leftrightarrow \forall C \in T_\Sigma (x): C[u] \in L \Leftrightarrow C[v] \in L$.\\

Beispiel: Betrachte alle B\"aume der Form \Tree [.f [.f [.... [.f a b ] ] b ] b ]

$L = \{ f(f(\dots f(a, b), b) \dots , b) \}$ \\

Dann gilt:  \Tree [.f a b ] $\equiv$ \Tree [.f [.f a b ] b ] \\

$C = $ \Tree [.f x b ],
$C' = $ \Tree [.f [.f x b ] b ],
$C'' = $ \Tree [.f a x ]

\subsection{Theorem (Myhill-Nerode)}

Die folgenden Aussagen sind \"aquivalent:\\

a) L ist erkennbar

b) L ist die Vereinigung von \"Aquivalenzklassen einer Kongruenz mit endlichem Index

c) $\equiv_L$ hat endlichen Index \\

Beweis:\\
\glqq a $\Rightarrow$ b \grqq:
Sei $\mathcal{A}$ vollst\"andiger DTA mit $L(\mathcal{A}) = L$.
Sei $\mathcal{A} = (Q, \Sigma, F, \Delta)$.\\
Definiere $u \equiv_\mathcal{A} v \Leftrightarrow \delta (u) = \delta (v)$.

Offensichtlich hat $\equiv_\mathcal{A}$ h\"ochstens $|Q|$-viele \"Aquivalenzklassen.
Au\ss erdem ist $\equiv_\mathcal{A}$ eine Kongruenz.
Nun ist $L$ Vereinigung aller Klassen $[u]_{\equiv_\mathcal{A}}$ mit $\delta(u) \in F$.\\

\glqq b $\Rightarrow$ c \grqq:
Sei $\sim $ eine Kongruenz mit dnlichem Index. Sei $u \sim  v$.
Wegen Lemma 1.20 gilt \\
$C[u] \sim  C[v] \forall C \in T_\Sigma (x)$.
Nun ist aer $L$ die Vereinigung von \"Aquivalenzklassen von $\sim $, das hei\ss t $C[u] \in L \Leftrightarrow C[v] \in L$.
Insbesondere ist also $u \equiv_L v$\\
Wir haben gezeigt: $v \in [u]_\sim  \Rightarrow v \in [u]_{\equiv_L}$, also $[u]_\sim  \leq [u]_{\equiv_L}$\\
(Also ist $\sim $ eine Verfeinerung von $\equiv_L$\\
Insbesondere hat $\equiv_L$ kleinern Index als $\sim $, also endlichen.\\

\glqq c $\Rightarrow$ a \grqq:
Die Zust\"ande $Q_\text{min}$ sind die \"Aquivalenzklassen bez\"uglich $\equiv_L$.
(Damit ist $Q_\text{min}$ endlich). Wir definieren Regeln\\
$f([u_1], \dots, [u_n]) \rightarrow [f(u_1, \dots, u_n)]$.\\
Das ist wohldefiniert, weil $\equiv_L$ eine Kompetenz ist.
Finalzust\"ande $F_\text{min}$ sind $\{ [u]_{\equiv_L} | u \in L \}$.\\
Dann akzeptiert $\mathcal{A}_\text{min} = (Q_\text{min}, \Sigma, F_\text{min}, \Delta_\text{min})$ die Sprache $L$.\\

Beispeiel:\\
Betrachte $\Sigma = \{a^{(0)}, g^{(1)}, f^{(2)} \}$ und \\
$L = \{ f(g^i(a), g^i(a)) | i \geq 0\}$\\

Betrachte $g^i(a)$ und $g^j(a)$ mit $i \neq j$.
Dann ist $C^i = f(x, g^i(a))$ ein Kontext mit $C^i[g^i(a)] \in L$, aber $C^i[g^j(a)] \notin L$.
Da ex unenlich viele $g^i(a)$ gibt, hat die Kongruenz bez\"uglich $L$ unendlichen Index, also ist $L$ nicht erkennbar.

\subsection{Korollar}

Ist $L$ erkennbar, gibt es einen bis auf Umbenennung der Zust\"ande eindeutigen, vollst\"andigen DBA $\mathcal{A}$ mit $L = L(\mathcal{A})$.
Dieser ist $\mathcal{A}_\text{min}$ aus obigem Beweis.\\
Beweis:\\
Sei $L = L(\mathcal{A})$. Vorher gesehen:\\
$\equiv_\mathcal{A}$ ist Verfeinerung von $\equiv_L$\\
Also ist $|Q| \geq |Q_\text{min}|$. Wir nehmen OBDA an:
beide reduziert. Sei nun $q in Q$. Getrachte ein $u \in T_\Sigma$ mit $\delta(u) = q$.
Betrachte die Funktion $\rho : Q \to Q_\text{min}$ mit $\delta(u) = q \mapsto \delta_\text{min}(u)$\\
Die Abbildung $\rho$ ist wohldefiniert, denn falls $\delta (u) = \delta (v)$, dann 
$u \equiv_\mathcal{A} v \Rightarrow u \equiv_L v \Leftrightarrow \delta_\text{min} (u) = \delta_\text{min} (v)$.
Au\ss erdem ist $\rho$ surjektiv, denn $\delta_\text{min} (u)$ hat das Urbild $\delta(u)$.\\
Also: $|Q| = |Q_\text{min}| \Rightarrow \rho$ ist Bijektion. $\Box$

\end{document}

