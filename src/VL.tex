
\documentclass[titlepage]{article}

%Paket-Liste
\usepackage{geometry}
\usepackage[german]{babel}
\usepackage{mathtools}
\usepackage{qtree}
\usepackage{amssymb}
\usepackage{amsmath}
\usepackage{tikz}
\usetikzlibrary{automata,positioning}

\setcounter{section}{-1}
\setlength{\parindent}{0pt}


\begin{document}

\title{\huge{Vorlesung Baumautomaten (Mitschrift)}}
\author{Benedikt El\ss mann (3720358)\\be57xocu@studserv.uni-leipzig.de\\\\Universit\"at Leipzig}

\maketitle

\thispagestyle{empty}

\tableofcontents

\newpage

\section{Einleitung}

Automaten lesen W\"orter $ w = a_1 \dots a_n $ und geben ''accept'' aus oder nicht.\\
Daf\"ur gibt es Erweiterungen, wie etwa:

\begin{itemize}
	\item gewichtete Automaten, das hei\ss t der Output ist ein Semiringelement
	\item Automaten mit Ged\"achtnis (Stack)
	\item Automaten \"uber anderen  Strukturen

	\begin{itemize}
		\item $\omega$-W\"orter $ w = a_1 \dots a_n $ 
		\item Graphen
		\item B\"aume\\
		\item Kombinationen dieser
	\end{itemize}

\end{itemize}

Typische Fragestellungen:

\begin{itemize}
	\item Ausdrucksst\"arke
	\item Darstellung als rationale Ausdr\"ucke (Kleene)
	\item Darstellung als Grammatik
	\item Darstellung als Logik
\end{itemize}

\section{B\"aume und Baumautomaten}

Wir betrachten \"uber $ A = \{a,b\} $ den Automaten $ \mathcal{A} $:

\begin{tikzpicture}[shorten >= 1pt, node distance=2cm, on grid, auto]
	\node[state, initial] (q0) {$ q_0 $};
	\node[state] (q1) [right=of q0 ] {$ q_1 $};
	\node[state, accepting] (q2) [right=of q1 ] {$ q_2 $};

	\path[->]	(q0)	edge 					node	{a}	(q1);
	\path[->]	(q0)	edge 	[loop above] 	node 	{b}	();
	\path[->]	(q1)	edge					node 	{b}	(q2);
	\path[->]	(q2)	edge	[loop above]	node	{a}	();
\end{tikzpicture}

mit $ L(\mathcal{A}) = b^\ast ab a^\ast $ .

Betrachtung des Wortes $ w = baba \in L(\mathcal{A}) $:\\

Der eindeutige erfolgreiche Lauf f\"ur $ w $ l\"asst sich darstellen als:\\

$q_0 baba \rightarrow b q_0 aba \rightarrow ba q_1 ba \rightarrow bab q_2 a \rightarrow baba q_2 \in F$ (Finalzustand)\\

Baumautomaten funktionieren analog. Unser erstes Beispiel wird

\Tree [.f [.f a b ] [.b ] ]

Akzeptiert mit dem Lauf:

\Tree [.f [.f a b ] [.b ] ] $\rightarrow$
\Tree [.f [.f $q_a$ $q_b$ ] [.$q_a$ ] ] $\rightarrow$
\Tree [.f [.$q_a$ $q_a$ $q_b$ ] [.$q_a$ ] ] $\rightarrow$
\Tree [.$q_f$ [.$q_a$ $q_a$ $q_b$ ] [.$q_a$ ] ]\\

mit $q_f \in F$ 

\subsection{Definition Rangalphabet}

Ein paar $(\Sigma, rk)$, wobei $\Sigma$ eine endliche Menge von Symbolen und $rk : \Sigma \rightarrow \mathbb{N}$ eine Abbildung ist, hei\ss t Rangalphabet.\\
F\"ur $f\in \Sigma$ hei\ss t $rk(f)$ der Rang (oder die Stelligkeit) von $f$.\\ \\
Intuitiv: $rk(f)$ ist die Anzhal der Kinder von $f$ in einem Baum.\\
Insbesondere ist die Anzahl der Kinder f\"ur jedes Symbol fest.\\ \\
Gilt $rk(f) = n$, schreiben wir auch $f^{(n)}$ statt $f$. wir schreiben:

\begin{itemize}
	\item 0-stellige Symbole (Konstanten) $a, b, \dots$
	\item un\"ar, bin\"ar, \dots $f, g, \dots$\\
\end{itemize}

Wir setzen $\Sigma^{(n)} = \{f \in \Sigma | rk(f) = n\}$\\

In 
\Tree [.f [.f a b ] [.b ] ] 
ist also $rk(f) = 2, rk(b) = 0$\\

\subsection{Definition Term, Tree}

Sei $(\Sigma, rk)$ ein Rangalphabet. Die Menge $T_{\Sigma}$ der B\"aume \"ueber $\Sigma$ ist induktiv definiert durch:

\begin{itemize}
	\item $\Sigma^0 \subseteq T_{\Sigma}$
	\item $f^{(n)} \in \Sigma$ . $t_1, \dots, t_n \in T_{\Sigma}$, dann ist $f(t_1, \dots, t_n) \in T_{\Sigma}$
\end{itemize}

Intuitiv sind $t_1, \dots, t_n$ die Kinder von $f$.\\
Z.B. ist \Tree [.f [.f a b ] [.b ] ] der Term $f(f(a,b),b)$.

\subsection{Definition H\"ohe}

Sei $(\Sigma, rk)$ ein Rangalphabet. Die H\"ohe $ht$ ist gegeben durch:

\begin{itemize}
	\item f\"ur $a^{(0)} \in \Sigma: ht(a) = 1$.
	\item f\"ur $f(t_1, \dots , t_n) \in T_{\Sigma} : ht(f) = 1 + max\{ ht(t_i) | i \in \{i, \dots, n\}\}$
\end{itemize}

Ziel: Zugriff auf einen Knoten innterhalb eines Baumes und deren Label.\\
Daf\"ur ordenen wir den Knoten Positionen zu. Das geht induktiv wie foelgt:

\subsection{Definition Position}

Sei $(\Sigma, rk)$ ein Rangalphabet. Die Positionenmenge ist definiert durch:

\begin{itemize}
	\item f\"ur $a^{(0)} \in T_{\Sigma}$ ist $Pos(a) = \{\varepsilon\}$
	\item f\"ur $f(t_1, \dots, t_n) \in T_{\Sigma}$ ist $Pos(f(t_1, \dots, t_n)) = \{\varepsilon\}1 \cdot Pos(t_1) \cup \dots \cup n \cdot Pos(t_n)$
\end{itemize}

Beispiel:\\
Betrachtung von $f(f(a,b),b)$ bzw. \Tree [.f [.f a b ] [.b ] ] der Term $f(f(a,b),b)$\\

$Pos(f) = \{\varepsilon, 1, 2, 1.1, 1.2\}$

\subsection{Definition der Label an den Positionen}

F\"ur einen Term der Form $t = f(t_1, \dots, t_n)$ ist das Symbol $t(p)$ in $t$ an p-ter Position induktiv definert durch:

\begin{itemize}
	\item $t(\varepsilon) = f$
	\item $t(ip) = t_i(p), i\in \{1, \dots , n\}$
\end{itemize}

Beispiel: Betrachtung von $f(f(a,b),b)$\\
Dann ist\\
$t(\varepsilon) = f$\\
$t(1) = t(1 \cdot \varepsilon) = t_1(\varepsilon) = f$\\
$t(2) = t(2 \cdot \varepsilon) = t_2(\varepsilon) = b$\\
$t(1.1) = t_1(1) = a$\\
$t(1.2) = t_2(1) = b$

\subsection{Definition Sub-Baum}

F\"ur $T_{\Sigma}$ ist ein Sub-Baum $t_{|p}$ an p-ter Position wie folgt definiert:

\begin{itemize}
	\item $Pos(t_{|p}) = \{ i | pi \ \in Pos(t)\}$
	\item $\forall q \in Pos(t_{|p}$ ist $t_{|p}(q) = t(pq)$
\end{itemize}

Wir schreiben $t[u]_p$ f\"ur den Baum, der entsteht, wenn man in $t$ den sub-Baum $t_{|p}$ durch $n$ ersetzt.\\

Beispiel: $f(f(a,b),b)$ bzw. \Tree [.f [.f a b ] [.b ] ]

$t_{|1} = f(a,b)$ \Tree [.f a b ]\\

$t_{|2} = t(1.2) = b$\\

$u = g(b,a)$ \Tree [.g b a ], dann ist\\

$t[u]_{|1} = f(g(b,a),b)$ \Tree [.f [.g b a ] [.b ] ]

\subsection{Definition Baumautomat}

Ein Buamautomat $ \mathcal{A}$ ist ein 4-Tupel $(Q,\Sigma,F,\Delta)$, wobei:\\
$Q \dots$ endliche Menge an Zus\"anden\\
$\Sigma \dots$ Rangalphabet, wobei $\Sigma \cup Q \neq \emptyset$\\
$F \dots \subseteq Q$ Finalzust\"ande\\
$\Delta \dots$ Menge von Regeln\\

$r: f(q_1 \dots q_n) \rightarrow q$\\
f\"ur $q, q_1, \dots , q_n \in Q$, f\"ur $a^{(0)} \in T_{\Sigma} : a \rightarrow q$\\

Beispiel:\\
$\mathcal{A} = \{\{q_a, q_b, q_f\},\{a^{(0)},b^{(0)},f^{(2)}\},\{q_f\}, \Delta\}$ \\
mit $\Delta = \{ a \rightarrow q_a, b \rightarrow q_b, f(q_a,q_b) \rightarrow q_a, f(q_a, q_b), f(q_a,q_b) \rightarrow q_f\}$

\subsection{Definition Lauf/Run}

Sei $\mathcal{A} = (Q, \Sigma, F, \Delta)$ ein Baumautomat und $t\in T_{\Sigma}$. 
Ein Lauf r f\"ur $t$ von $\mathcal{A}$ ist ein Term mit

\begin{itemize}
	\item $Pos(r) = Pos(t)$
	\item Ist $t(p) = a$ ein Blatt, dann ist $r(p) = q_a$, nur wenn $(a\rightarrow q_a) \in \Delta$ 
	\item Ist $t(p)=f^{(m)}$, dann ist $r(p) = q$, wenn $(f(q_1, \dots, q_n) \rightarrow q) \in \Delta$ und $r(p_i) = q_i$, $i \in \{1, \dots, n\}$
\end{itemize}

Ein Lauf ist erfolgreich, wenn $r(\varepsilon) \in F$. Der Automat $\mathcal{A}$ akzeptiert $t$, falls es einen erfolgreichen Lauf
f\"ur $t$ von $\mathcal{A}$ gibt.\\
Wir bezeichnen mit $L(\mathcal{A}) = \{t \in T_{\Sigma} | \mathcal{A}$ akzeptiert $t\}$ die von $\mathcal{A}$ erkannte Baumsprache. Eine Sprache $L \subseteq T_{\Sigma}$ hei\ss t erkennbar, falls ein Baumautomat $\mathcal{A}$ existiert mit $L=L(\mathcal{A})$.\\

Um einzelne Schritte von Baumautomaten zu formalisieren, betrachten wir die \textit{move relation} $\rightarrow_{\mathcal{A}}$, 
definiert wie folgt:\\
Gegeben sei $\mathcal{A} = (Q, \Sigma, F, \Delta)$, dann ist $t \rightarrow _{\mathcal{A}} t'$ mit $t, t' \in T_{\Sigma \cup Q}$, falls

\begin{itemize}
	\item $t(p) = f^{(n)}$
	\item $t(pi) = q_i$ f\"ur $i \in \{1, \dots, n\}$ und $p_i$ sind Bl\"atter
	\item $(f(q_1, \dots, q_n) \rightarrow q) \in \Delta$
	\item und $t' = t[q]_p$
\end{itemize}

Mit $\rightarrow^\ast_{\mathcal{A}}$ bezeichnen wir die transitive H\"ulle von $\rightarrow_{\mathcal{A}}$.

\subsection{Lemma}

Sei $\mathcal{A} = (Q, \Sigma, F, \Delta)$ ein Baumautomat. Dann ist 
$L(\mathcal{A}) = \{t \in T_{\Sigma} | t \rightarrow^\ast_{\mathcal{A}} q$ mit $q \in F\} (=Z)$\\

Beweis: \glqq$L(\mathcal{A}) \subseteq Z$\grqq:\\

Wir zeigen: Es existiert ein Run $r$ f\"ur $t$ von $\mathcal{A}$ mit $r(\varepsilon) = q$, dann ist $t \rightarrow ^\ast_{\mathcal{A}} q$\\ \\
Inuktionsannahme:\\
$t=a^{(0)} \in T_{\Sigma}$. Dann gilt $a \in L({\mathcal{A}})$, falls ein Lauf $r$ existiert mit $r(a) = q_a$ und $(a \rightarrow q_a) \in \Delta$. Dann folgt $a \rightarrow^\ast_{{\mathcal{A}}} q_a$.\\
Sei nun $t=f(t_1, \dots, t_n)$\\ \\

Induktionsvoraussetzung:\\
Falls f\"ur $t_1, \dots , t_n$ L\"aufe $r_i$ existieren mit $r_i(\varepsilon) = q_i$, dann gilt auch $t_i \rightarrow ^\ast_{\mathcal{A}} q_i$ mit $i\in \{1, \dots , n\}$\\ \\

Induktionsschritt:\\
zu zeigen: Es existiert ein Lauf $r$ f\"ur $t$ mit $r(\varepsilon) = q$, dann $t \rightarrow ^\ast_\mathcal{A} q$.\\
Sei also $r$ ein Lauf mit $r(\varepsilon) = q$. Dann ist $r(i) = q_i$, $i \in \{1, \dots, n\}$, mit
$(f(q_1, \dots, q_n) \rightarrow q) \in \Delta$. Laut Induktionsvoraussetzung gilt nun,
$t_i \rightarrow ^\ast_\mathcal{A} q_i$, $i \in \{1, \dots, n\}$.\\
Damit $t = f(t_1, \dots, t_n) 
\rightarrow ^\ast_\mathcal{A} f(q_1, t_2, \dots, t_n)
\rightarrow ^\ast_\mathcal{A} \dots
\rightarrow ^\ast_\mathcal{A} f(q_1, \dots, q_n)$\\
Des weiteren haben wir die regel $f(q_1, \dots, q_n) \rightarrow q$, das hei\ss t $f(q_1, \dots, q_n) \rightarrow ^\ast_\mathcal{A} q$.\\ \\

Insgesamt also $t \rightarrow ^\ast_\mathcal{A} q$ \\

Beweis: \glqq$L(Z \subseteq \mathcal{A})$\grqq: analog\\ \\

Einige Beispiele f\"ur Baumautomaten:\\

1. Sei $B = (\{q_0, q_1\}, \{0^{(0)},1^{(0)},\lnot^{(1)},\land^{(2)},\lor^{(2)}\,\{q_1\},\Delta\}$ mit\\
$\Delta = \{ 0 \rightarrow q_0, 1 \rightarrow q_1,\\
\lnot (q_0) \rightarrow q_1, \lnot (q_1) \rightarrow q_0,\\
\land (q_0, q_0) \rightarrow q_0, \land (q_0, q_1) \rightarrow q_0, \land (q_1, q_0) \rightarrow q_0, \land (q_1, q_1) \rightarrow q_1\\
\lor (q_0, q_0) \rightarrow q_0, \lor (q_0, q_1) \rightarrow q_1, \lor (q_1, q_0) \rightarrow q_1, \lor (q_1, q_1) \rightarrow q_1\}$\\

Beispiellauf:\\

\Tree [.$\lnot$ [.$\land$ [.$\lor$ 0 [.$\lnot$ 1 ] ] [.$\lor$ 1 [.$\land$ 0 0 ] ] ] ] $\rightarrow$
\Tree [.$\lnot$ [.$\land$ [.$\lor$ $q_0$ [.$\lnot$ $q_1$ ] ] [.$\lor$ $q_1$ [.$\land$ $q_0$ $q_0$ ] ] ] ] $\rightarrow$
\Tree [.$\lnot$ [.$\land$ [.$\lor$ $q_0$ [.$q_0$ ] ] [.$\lor$ $q_1$ [.$\land$ $q_0$ $q_0$ ] ] ] ] $\rightarrow$ \\ \\
\Tree [.$\lnot$ [.$\land$ [.$\lor$ $q_0$ $q_0$ ] [.$\lor$ $q_1$ $q_0$ ] ] ] $\rightarrow$
\Tree [.$\lnot$ [.$\land$ $q_0$ $q_1$ ] ] $\rightarrow$
\Tree [.$\lnot$ $q_0$ ] $\rightarrow$
\Tree [.$q_1$ ] \\

2. $(a^nb^n light)$\\

Betrachten $\mathcal{A} = (\{q_a, q_b, q_f\}, \{a^{(0)}, b^{(0)}, f^{(3)}, g^{(2)}\}, \{q_f\}, \Delta)$\\
mit $\Delta = \\
\{a \rightarrow q_a, b \rightarrow q_b, g (q_a, q_b) \rightarrow q_f, f (q_a, q_f, q_b) \rightarrow q_f\}$\\

Beispiellauf:\\

\Tree [.$f$ $a$ [.$f$ $a$ [.$g$ $a$ $b$ ] $b$ ] $b$ ] $\rightarrow$
\Tree [.$f$ $q_a$ [.$f$ $q_a$ [.$g$ $q_a$ $q_b$ ] $q_b$ ] $q_b$ ] $\rightarrow$
\Tree [.$f$ $q_a$ [.$f$ $q_a$ $q_f$ $q_b$ ] $q_b$ ] $\rightarrow$
\Tree [.$f$ $q_a$ $q_f$ $q_b$ ] $\rightarrow$
\Tree [.$q_f$ ] \\

$\mathcal{A}$ akzeptiert also alle B\"aume der Form:\\
\Tree [.$f$ $a$ [.$f$ $a$ [.... [.$g$ $a$ $b$ ] ] $b$ ] $b$ ]\\

3. Simulation eines Wortautomaten: (siehe \"Ubung)\\

%2 weitere Seiten vom 10. April

\end{document}

